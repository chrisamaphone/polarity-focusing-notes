\documentclass{article}
\usepackage{graphicx} % Required for inserting images
\usepackage{proof-dashed,amsmath,amssymb,amsthm}
\usepackage{xcolor}
\usepackage{stmaryrd}
\usepackage{microtype}
\usepackage[utf8]{inputenc}
\newcommand{\exif}[1]{\mathsf{if}\;{#1}}
\newcommand{\exthen}[1]{\ \mathsf{then}\;{#1}}
\newcommand{\exelse}[1]{\ \mathsf{else}\;{#1}}
\newcommand{\exite}[1]{\mathsf{ite}(#1)}
\newcommand{\explus}[1]{\mathsf{plus}(#1)}
\newcommand{\extrue}{\mathsf{true}}
\newcommand{\exfalse}{\mathsf{false}}
\newcommand{\exnum}[1]{\mathsf{num}(#1)}
\newcommand{\exbool}[1]{\mathsf{bool}(#1)}
\newcommand{\exlet}[1]{\mathsf{let}(#1)}
\newcommand{\exfun}[1]{\ensuremath{\lambda{#1}}}
% \newcommand{\exapp}[2]{\mathsf{app}({#1}, {#2})}
\newcommand{\exapp}[2]{{#1}{(#2)}}
\newcommand{\exunit}{()}
\newcommand{\expair}[1]{(#1)}
\newcommand{\exlpair}[1]{\langle #1 \rangle}
\newcommand{\exinl}[1]{\mathsf{in}_1\,{#1}}
\newcommand{\exinr}[1]{\mathsf{in}_2\,{#1}}
%\newcommand{\expil}[1]{\ensuremath{\pi_1}\,{#1}}
%\newcommand{\expir}[1]{\ensuremath{\pi_2}\,{#1}}
% "dot" syntax:
\newcommand{\expil}[1]{{#1}.1}
\newcommand{\expir}[1]{{#1}.2}
\newcommand{\excase}[1]{\mathsf{case}(#1)}
\newcommand{\exsplit}[1]{\mathsf{split}(#1)}
\newcommand{\exlam}[1]{\exfun{#1}}
\newcommand{\exzero}{\mathsf{zero}}
\newcommand{\exsucc}[1]{\mathsf{succ}(#1)}
\newcommand{\exrec}[1]{\mathsf{rec}(#1)}
\newcommand{\exdiverge}{\bot}
\newcommand{\exfix}[2]{\mathsf{fix}_{#1}(#2)}
\newcommand{\exifz}[1]{\mathsf{ifz}(#1)}
\newcommand{\extyab}[1]{\Lambda{#1}}
\newcommand{\extyapp}[2]{#1[#2]}
\newcommand{\exfail}{\mathsf{fail}}
\newcommand{\extry}[1]{\mathsf{trycatch}(#1)}
\newcommand{\exalloc}[1]{\mathsf{alloc}(#1)}
\newcommand{\exassign}[2]{{#1} := {#2}}
\newcommand{\exderef}[1]{?{#1}}

% types
\newcommand{\tybool}{\mathsf{Bool}}
\newcommand{\tynum}{\mathsf{Num}}
\newcommand{\tynat}{\mathsf{nat}}
\newcommand{\tyzero}{\mathbf{0}}
% \newcommand{\tyone}{\mathbb{1}}
\newcommand{\tyone}{\mathbf{1}}
% \newcommand{\tytwo}{\mathbb{2}}
\newcommand{\tytwo}{\mathbf{2}}
\newcommand{\typrod}{\times}
\newcommand{\tyfun}{\to}
\newcommand{\tysum}{+}
\newcommand{\tywith}{\;\&\;}
\newcommand{\typarfun}{\rightharpoonup}
\newcommand{\tyall}{\forall}
\newcommand{\tyref}[1]{\mathsf{ref}(#1)}
\newcommand{\tylolli}{\multimap}
\newcommand{\tyoplus}{\oplus}
\newcommand{\tytens}{\otimes}
\newcommand{\tytop}{\top}
\newcommand{\tybang}{\,!}

\newcommand{\pzero}{\mathsf{zero}}
\newcommand{\psucc}{\mathsf{succ}}

\newcommand{\evals}{\Downarrow}
\newcommand{\isval}{\ \mathsf{value}}
\newcommand{\istype}{\ \mathsf{type}}
\newcommand{\isctx}{\ \mathsf{ctx}}

\newcommand{\steps}{\mapsto}
\newcommand{\stepstar}{\mapsto^{*}}
\newcommand{\reduces}{\longrightarrow}

% Stacks
\newcommand{\seval}{\triangleright}
\newcommand{\sret}{\triangleleft}
\newcommand{\hole}{\square}
%\DeclareUnicodeCharacter{1F635}{\failstate}
%\DeclareRobustCommand\failstate{%
%  \unskip\nobreak\thinspace\textemdash\allowbreak\thinspace\ignorespaces}
\newcommand{\failstate}{\blacktriangleleft}

%% Program equivalence
\newcommand{\iso}{\simeq}
% \newcommand{\eequiv}{\cong}
\newcommand{\eequiv}{\approx}
\newcommand{\vequiv}{\sim}

% Semantics
\newcommand{\mathnat}{\mathbf{N}}
\newcommand{\encode}[1]{\ulcorner #1\urcorner}
\newcommand{\vsem}[1]{\llbracket #1 \rrbracket}
\newcommand{\rel}[1]{\mathbf{Rel}(#1)}

% Symbol reference: https://milde.users.sourceforge.net/LUCR/Math/mathpackages/amssymb-symbols.pdf

% Ordered logic
\newcommand{\ordMult}{\bullet}
\newcommand{\ordUnit}{\epsilon}
\newcommand{\ordArrL}{\rightarrowtail}
\newcommand{\ordArrR}{\twoheadrightarrow}
\newcommand{\ordCtx}{\Omega}
\newcommand{\gnab}{\text{\textexclamdown}}

% Linear logic
\newcommand{\linMult}{\otimes}
\newcommand{\linUnit}{\mathbb{1}}
\newcommand{\linArr}{\multimap}
\newcommand{\linWith}{\&}
\newcommand{\linTop}{\top}
\newcommand{\linPlus}{\oplus}
\newcommand{\linOne}{\mathbb{1}}
\newcommand{\linZero}{\mathbb{0}}
\newcommand{\linCtx}{\Delta}
\newcommand{\bang}{!}

% Intuitionistic logic
\newcommand{\iand}{\wedge}
\newcommand{\ior}{\vee}
\newcommand{\imp}{\supset}
\newcommand{\itrue}{\top}
\newcommand{\ifalse}{\bot}

% Shifts
\newcommand{\ushift}[2]{\uparrow\smash{{}^{#2}_{#1}}}
\newcommand{\dshift}[2]{\downarrow\smash{{}^{#1}_{#2}}}

% Judgments
\newcommand{\istrue}{\ \mathsf{true}}
\newcommand{\proves}{\Rightarrow}
\newcommand{\entails}{\vdash}
\newcommand{\hyp}{\ \mathsf{hyp}}
\newcommand{\conc}{\ \mathsf{conc}}
\newcommand{\verif}{\textcolor{blue}{\,\uparrow}}
\newcommand{\use}{\textcolor{red}{\,\downarrow}}

\newcommand{\DD}{\mathcal{D}}
\newcommand{\EE}{\mathcal{E}}
\newcommand{\FF}{\mathcal{F}}


\usepackage{aliascnt}

\newtheorem{theorem}{Theorem}

\newaliascnt{conjecture}{theorem}
\newtheorem{conjecture}[conjecture]{Conjecture}
\aliascntresetthe{conjecture}
\providecommand*{\conjectureautorefname}{Conjecture}
% \newtheorem{conjecture}[theorem]{Conjecture}

\newaliascnt{lemma}{theorem}
\newtheorem{lemma}[lemma]{Lemma}
\aliascntresetthe{lemma}
\providecommand*{\lemmaautorefname}{Lemma}
% \newtheorem{lemma}[theorem]{Lemma}

\newaliascnt{corollary}{theorem}
\newtheorem{corollary}[corollary]{Corollary}
\aliascntresetthe{corollary}
\providecommand*{\corollaryautorefname}{Corollary}
% \newtheorem{corollary}[theorem]{Corollary}

\newtheorem{exercise}{Exercise}
\providecommand*{\exerciseautorefname}{Exercise}

\newtheorem{discuss}{Discussion Question}
\providecommand*{\exerciseautorefname}{Discussion Question}

\newtheorem{definition}[theorem]{Definition}

% \newenvironment{proof}{\trivlist \item[\hskip \labelsep{\bf 
% Proof:}]}{\hfill$\Box$ \endtrivlist}
\newenvironment{sketch}{\trivlist \item[\hskip \labelsep{\bf 
Proof sketch:}]}{\hfill$\Box$ \endtrivlist}
\newenvironment{attempt}{\trivlist \item[\hskip \labelsep{\bf 
Proof attempt:}]}{\hfill$\Diamond$ \endtrivlist}

\title{Lecture 3: Sequent Calculus}
\author{Chris Martens}
\date{\today}

\begin{document}

\maketitle

\section{Introduction}

Lecture outline:
\begin{itemize}
    \item Sequent calculus: left and right rules
    \item Statement of Cut and Identity properties
    \item Proof: Admissibility of Cut 
    \item Proof: ND into sequent calculus (via Cut)
    \item Identity
\end{itemize}

This paper sources extensively from (TODO cite) Gentzen'35
and Frank Pfenning's 2017 lecture notes on Sequent Calculus and
Cut Elimination.

\section{Sequent calculus: left and right rules for connectives}

The calculus presented herein corresponds approximately to Gentzen's LJ.

We will write the judgment as $\Gamma \proves A$,
where $\Gamma$ is an unordered list of propositions
$A_1,\ldots,A_n$ and $A$ is a proposition.
To be more precise, we can think of the assumptions
in $\Gamma$ as judgments of the form $A\hyp$
and the conclusion as a judgment of the form
$A\conc$. However, this treatment represents a
divergence from Gentzen, who treated the sequent
$\Gamma \proves A$ as a syntactic structure.

For each connective, we will define {\em left rules}
that define a proposition's meaning on the left side
of the sequent and {\em right rules} that determine
its meaning on the right.

Conjunction:
\[
  \infer[\iand R]
  {\Gamma \proves A \iand B}
  {\Gamma \proves A
  &
  \Gamma \proves A}
\qquad
  \infer[\iand L_1]
  {\Gamma, A \iand B \proves C}
  {\Gamma, A \iand B, A \proves C}
\qquad
  \infer[\iand L_2]
  {\Gamma, A \iand B \proves C}
  {\Gamma, A \iand B, B \proves C}
\]

Disjunction:
\[
  \infer[\ior R_1]
  {\Gamma \proves A \ior B}
  {\Gamma \proves A}
\qquad
  \infer[\ior R_2]
  {\Gamma \proves A \ior B}
  {\Gamma \proves B}
\]
\[
  \infer[\ior L]
  {\Gamma, A \ior B \proves C}
  {\Gamma, A \ior B, A \proves C
   &
   \Gamma, A \ior B, B \proves C
  }
\]

Implication:
\[
  \infer[\imp R]
  {\Gamma \proves A \imp B}
  {\Gamma, A \proves B}
\qquad
  \infer[\imp L]
  {\Gamma, A \imp B \proves C}
  {\Gamma, A \imp B \proves A
  &
  \Gamma, A \imp B, B \proves C}
\]


Truth and Falsehood:
\[
  \infer[\top R]
  {\Gamma \proves \top}{}
  \qquad
  \mathrm{(no\ \top L)}
  \qquad
  \mathrm{(no\ \bot R)}
  \qquad
  \infer[\bot L]
  {\Gamma, \bot \proves C} {}
\]

Negation $\neg A$ is, as before,
defined as $A \imp \bot$.

Finally, we need something analogous to the
hypothesis rule from natural deduction
in order to make use of things on the left
of the sequent on the right. We call this the
``identity rule'' and sometimes refer to its
conclusion as an ``initial sequent.''

\[
  \infer[\mathsf{id}]
  {\Gamma, A \proves A}
  {}
\]


\subsection{Proof Examples}

We will go through some subset of the following as examples
in class.
\begin{itemize}
  \item $A \ior B \imp A \ior B$
  \item $(A \imp B) \iand A \imp B$
  \item $(A \imp (B \ior C)) \imp (A \iand \neg B) \imp C$
  \item $(A \ior B) \iand C \imp (A \iand C) \ior (B \iand C)$
\end{itemize}

\begin{exercise}
  Typeset proofs of $\cdot \proves A$
  for each formula $A$ above.
\end{exercise}

\section{Observations}

\subsection{Structural Rules}

Some presentations of sequent calculus include
the so-called ``structural rules'', weakening (or thinning) 
and contraction.

\[
  \infer[\mathsf{wk}]
  {\Gamma, A \proves C}
  {\Gamma \proves C}
  \qquad
  \infer[\mathsf{contr}]
  {\Gamma, A \proves C}
  {\Gamma, A, A \proves C}
\]

However, it will be to our advantage to prove these rules {\em admissible}
instead of explicitly including them.

Proving a rule admissible means: assume there is some derivation for
premises, and show that there is a derivation of the conclusion.

We can come up with a theorem statement corresponding to the rule to show
it admissible.

\begin{theorem}[Weakening]
  If $\Gamma \proves C$, then $\Gamma, A \proves C$ with
  a structurally identical deduction.
\end{theorem}

\begin{proof}
  Add $A$ to $\Gamma$ in every sequent appearing in 
  the given derivation of $\Gamma \proves C$. The rules still apply in each
  case since they are parametric in $\Gamma$. 
  The result is a structurally identical derivation
  of $\Gamma, A \proves C$.
\end{proof}

(Note that we could make the above more formal by induction over the
structure of the derivation, but it would be a very boring proof.)

\begin{exercise} 
  State and prove a corresponding theorem for Contraction.
  If you are unable to make the proof rigorous,
  discuss why and what might be done to address it.
\end{exercise}

\subsection{Non-provability}

In sequent calculus, it is easy to demonstrate
that certain sequents are {\em not} provable.
We will work through some examples:

\begin{itemize}
  \item Soundness: no proof of $\cdot \proves \bot$
  \item Disjunction property: if $\cdot \proves A \ior B$,
    then $\cdot \proves A$ or $\cdot \proves B$.
  \item Non-provability of excluded middle (LEM): there is no proof
    of $\cdot \proves A \ior \lnot A$ for arbitrary $A$.
  \item Non-provability of double-negation elimination (DNE):
    there is no proof of $\cdot \proves \lnot \lnot A \implies A$.
\end{itemize}

\begin{exercise}
  Typeset proofs of the above non-provability arguments.
\end{exercise}

\begin{exercise}
  Find another classically valid, but not intuitionistically valid,
  proposition $A$ and demonstrate that $\cdot \proves A$ is not provable
  in general.
\end{exercise}

\begin{exercise}
  Why do we keep qualifying each of these statements with ``in general''
  or ``for arbitrary $A$''? Explain how the situation changes
  when we are allowed to talk about specific propositions $A$.
\end{exercise}


\section{Sequent Calculus Metatheory: Cut and Identity}

There are two theorems corresponding to internal soundness and
completeness for sequent calculus.

The soundness theorem is {\em cut}:

\begin{proposition}[Cut]
  If $\Gamma \proves A$ and $\Gamma, A \proves C$ then $\Gamma \proves C$.
\end{proposition}

\newcommand{\provesid}{\proves^{\mathsf{id}}}

The completeness theorem requires that we 
restrict the identity rule to atomic propositions.
We will call this restricted form of provability
$\provesid$.

\[
  \infer[\mathsf{id}]
  {\Gamma, P \provesid P}
  {}
\]

$P$ here is atomic. All other rules in $\provesid$ are
identical to $\proves$.

\begin{proposition}[Identity]
  For all propositions $A$,
  $A \provesid A$.
\end{proposition}

\begin{exercise}
  Discuss why these metatheorem statements correspond to soundness and
  completeness for sequent calculus.
\end{exercise}

\section{Proof of Cut Admissibility}


\begin{theorem}[Cut]
  If $\Gamma \proves A$ and $\Gamma, A \proves C$ then $\Gamma \proves C$.
\end{theorem}

\begin{proof}
  By induction on the lexicographic ordering of $A$ followed
  by the unordered pair of $\DD$ and $\EE$, where $\DD$ is the derivation
  of the first assumption and $\EE$ is the derivation of the second.

  TODO add cases.
\end{proof}

\section{Adequacy of Sequent Calculus for Natural Deduction}

Statement: if $\Gamma \vdash A\istrue$ then $\ulcorner \Gamma \urcorner
\proves A$.

Proof: by induction on the structure of the derivation of $\Gamma \vdash A
\istrue$.



\bibliographystyle{plainnat}
\bibliography{main}

\end{document}
