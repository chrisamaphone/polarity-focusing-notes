\documentclass{article}
\usepackage{graphicx} % Required for inserting images
\usepackage{proof-dashed,amsmath,amssymb,amsthm}
\usepackage{xcolor}
\usepackage{stmaryrd}
\usepackage{microtype}
\usepackage[utf8]{inputenc}
\newcommand{\exif}[1]{\mathsf{if}\;{#1}}
\newcommand{\exthen}[1]{\ \mathsf{then}\;{#1}}
\newcommand{\exelse}[1]{\ \mathsf{else}\;{#1}}
\newcommand{\exite}[1]{\mathsf{ite}(#1)}
\newcommand{\explus}[1]{\mathsf{plus}(#1)}
\newcommand{\extrue}{\mathsf{true}}
\newcommand{\exfalse}{\mathsf{false}}
\newcommand{\exnum}[1]{\mathsf{num}(#1)}
\newcommand{\exbool}[1]{\mathsf{bool}(#1)}
\newcommand{\exlet}[1]{\mathsf{let}(#1)}
\newcommand{\exfun}[1]{\ensuremath{\lambda{#1}}}
% \newcommand{\exapp}[2]{\mathsf{app}({#1}, {#2})}
\newcommand{\exapp}[2]{{#1}{(#2)}}
\newcommand{\exunit}{()}
\newcommand{\expair}[1]{(#1)}
\newcommand{\exlpair}[1]{\langle #1 \rangle}
\newcommand{\exinl}[1]{\mathsf{in}_1\,{#1}}
\newcommand{\exinr}[1]{\mathsf{in}_2\,{#1}}
%\newcommand{\expil}[1]{\ensuremath{\pi_1}\,{#1}}
%\newcommand{\expir}[1]{\ensuremath{\pi_2}\,{#1}}
% "dot" syntax:
\newcommand{\expil}[1]{{#1}.1}
\newcommand{\expir}[1]{{#1}.2}
\newcommand{\excase}[1]{\mathsf{case}(#1)}
\newcommand{\exsplit}[1]{\mathsf{split}(#1)}
\newcommand{\exlam}[1]{\exfun{#1}}
\newcommand{\exzero}{\mathsf{zero}}
\newcommand{\exsucc}[1]{\mathsf{succ}(#1)}
\newcommand{\exrec}[1]{\mathsf{rec}(#1)}
\newcommand{\exdiverge}{\bot}
\newcommand{\exfix}[2]{\mathsf{fix}_{#1}(#2)}
\newcommand{\exifz}[1]{\mathsf{ifz}(#1)}
\newcommand{\extyab}[1]{\Lambda{#1}}
\newcommand{\extyapp}[2]{#1[#2]}
\newcommand{\exfail}{\mathsf{fail}}
\newcommand{\extry}[1]{\mathsf{trycatch}(#1)}
\newcommand{\exalloc}[1]{\mathsf{alloc}(#1)}
\newcommand{\exassign}[2]{{#1} := {#2}}
\newcommand{\exderef}[1]{?{#1}}

% types
\newcommand{\tybool}{\mathsf{Bool}}
\newcommand{\tynum}{\mathsf{Num}}
\newcommand{\tynat}{\mathsf{nat}}
\newcommand{\tyzero}{\mathbf{0}}
% \newcommand{\tyone}{\mathbb{1}}
\newcommand{\tyone}{\mathbf{1}}
% \newcommand{\tytwo}{\mathbb{2}}
\newcommand{\tytwo}{\mathbf{2}}
\newcommand{\typrod}{\times}
\newcommand{\tyfun}{\to}
\newcommand{\tysum}{+}
\newcommand{\tywith}{\;\&\;}
\newcommand{\typarfun}{\rightharpoonup}
\newcommand{\tyall}{\forall}
\newcommand{\tyref}[1]{\mathsf{ref}(#1)}
\newcommand{\tylolli}{\multimap}
\newcommand{\tyoplus}{\oplus}
\newcommand{\tytens}{\otimes}
\newcommand{\tytop}{\top}
\newcommand{\tybang}{\,!}

\newcommand{\pzero}{\mathsf{zero}}
\newcommand{\psucc}{\mathsf{succ}}

\newcommand{\evals}{\Downarrow}
\newcommand{\isval}{\ \mathsf{value}}
\newcommand{\istype}{\ \mathsf{type}}
\newcommand{\isctx}{\ \mathsf{ctx}}

\newcommand{\steps}{\mapsto}
\newcommand{\stepstar}{\mapsto^{*}}
\newcommand{\reduces}{\longrightarrow}

% Stacks
\newcommand{\seval}{\triangleright}
\newcommand{\sret}{\triangleleft}
\newcommand{\hole}{\square}
%\DeclareUnicodeCharacter{1F635}{\failstate}
%\DeclareRobustCommand\failstate{%
%  \unskip\nobreak\thinspace\textemdash\allowbreak\thinspace\ignorespaces}
\newcommand{\failstate}{\blacktriangleleft}

%% Program equivalence
\newcommand{\iso}{\simeq}
% \newcommand{\eequiv}{\cong}
\newcommand{\eequiv}{\approx}
\newcommand{\vequiv}{\sim}

% Semantics
\newcommand{\mathnat}{\mathbf{N}}
\newcommand{\encode}[1]{\ulcorner #1\urcorner}
\newcommand{\vsem}[1]{\llbracket #1 \rrbracket}
\newcommand{\rel}[1]{\mathbf{Rel}(#1)}

% Symbol reference: https://milde.users.sourceforge.net/LUCR/Math/mathpackages/amssymb-symbols.pdf

% Ordered logic
\newcommand{\ordMult}{\bullet}
\newcommand{\ordUnit}{\epsilon}
\newcommand{\ordArrL}{\rightarrowtail}
\newcommand{\ordArrR}{\twoheadrightarrow}
\newcommand{\ordCtx}{\Omega}
\newcommand{\gnab}{\text{\textexclamdown}}

% Linear logic
\newcommand{\linMult}{\otimes}
\newcommand{\linUnit}{\mathbb{1}}
\newcommand{\linArr}{\multimap}
\newcommand{\linWith}{\&}
\newcommand{\linTop}{\top}
\newcommand{\linPlus}{\oplus}
\newcommand{\linOne}{\mathbb{1}}
\newcommand{\linZero}{\mathbb{0}}
\newcommand{\linCtx}{\Delta}
\newcommand{\bang}{!}

% Intuitionistic logic
\newcommand{\iand}{\wedge}
\newcommand{\ior}{\vee}
\newcommand{\imp}{\supset}
\newcommand{\itrue}{\top}
\newcommand{\ifalse}{\bot}

% Shifts
\newcommand{\ushift}[2]{\uparrow\smash{{}^{#2}_{#1}}}
\newcommand{\dshift}[2]{\downarrow\smash{{}^{#1}_{#2}}}

% Judgments
\newcommand{\istrue}{\ \mathsf{true}}
\newcommand{\proves}{\Rightarrow}
\newcommand{\entails}{\vdash}
\newcommand{\hyp}{\ \mathsf{hyp}}
\newcommand{\conc}{\ \mathsf{conc}}
\newcommand{\verif}{\textcolor{blue}{\,\uparrow}}
\newcommand{\use}{\textcolor{red}{\,\downarrow}}

\newcommand{\DD}{\mathcal{D}}
\newcommand{\EE}{\mathcal{E}}
\newcommand{\FF}{\mathcal{F}}


\usepackage{aliascnt}

\newtheorem{theorem}{Theorem}

\newaliascnt{conjecture}{theorem}
\newtheorem{conjecture}[conjecture]{Conjecture}
\aliascntresetthe{conjecture}
\providecommand*{\conjectureautorefname}{Conjecture}
% \newtheorem{conjecture}[theorem]{Conjecture}

\newaliascnt{lemma}{theorem}
\newtheorem{lemma}[lemma]{Lemma}
\aliascntresetthe{lemma}
\providecommand*{\lemmaautorefname}{Lemma}
% \newtheorem{lemma}[theorem]{Lemma}

\newaliascnt{corollary}{theorem}
\newtheorem{corollary}[corollary]{Corollary}
\aliascntresetthe{corollary}
\providecommand*{\corollaryautorefname}{Corollary}
% \newtheorem{corollary}[theorem]{Corollary}

\newtheorem{exercise}{Exercise}
\providecommand*{\exerciseautorefname}{Exercise}

\newtheorem{discuss}{Discussion Question}
\providecommand*{\exerciseautorefname}{Discussion Question}

\newtheorem{definition}[theorem]{Definition}

% \newenvironment{proof}{\trivlist \item[\hskip \labelsep{\bf 
% Proof:}]}{\hfill$\Box$ \endtrivlist}
\newenvironment{sketch}{\trivlist \item[\hskip \labelsep{\bf 
Proof sketch:}]}{\hfill$\Box$ \endtrivlist}
\newenvironment{attempt}{\trivlist \item[\hskip \labelsep{\bf 
Proof attempt:}]}{\hfill$\Diamond$ \endtrivlist}

\title{Lecture 4: Relating Natural Deduction to Sequent Calculus via Cut
Admissibility}
\author{Chris Martens}
\date{\today}

\begin{document}

\maketitle

\section{Introduction}

Lecture outline:
\begin{itemize}
    \item Cut Admissibility proof
    \item Natural Deduction to Sequent Calculus via Cut Elimination
    \item Verifications and Uses
\end{itemize}

This lecture sources extensively from Frank Pfenning's 2017 lecture notes
on Sequent Calculus and Cut Elimination.

\section{Cut Admissibility}

Recall our statement of the Cut theorem.

\begin{theorem}[Cut]
  If $\Gamma \proves A$ and $\Gamma, A \proves C$ then $\Gamma \proves C$.
\end{theorem}

\begin{proof}
  By induction on the lexicographic ordering of $A$ followed
  by the unordered pair of $\DD$ and $\EE$, where $\DD$ is the derivation
  of the first assumption and $\EE$ is the derivation of the second.

  The cases for the proof fall into three categories:
  \begin{itemize}
    \item The so-called {\em principal cuts}, in which the cut formula 
      $A$ is the {\em principal formula} for the outermost rule of both 
      $\DD$ and $\EE$. 
      The principal formula of a rule is the one it matches against,
      e.g. $A \iand B$ in $\iand L$ and $\iand R$.
    \item The cut formula is not the principal formula of the outermost
      rule comprising $\DD$.
    \item The cut formula is not the principal formula of the outermost
      rule comprising $\EE$.
  \end{itemize}

  In class, we discuss three cases.
  \begin{itemize}
    \item Principal cut for $\imp$.
    \item Side formula cut (1) for $\ior L$.
    \item Side formula cut (2) for $\iand R$.
  \end{itemize}
\end{proof}

\section{From Natural Deduction to Sequent Calculus}

\begin{theorem}[ND to SC]
  \label{thm:nd-sc}
  If $\Gamma \vdash A\istrue$ then $\ulcorner \Gamma \urcorner
\proves A$.
  ($\ulcorner \Gamma \urcorner$ turns each $A_i\istrue \in \Gamma$
  into a bare $A_i$).
\end{theorem}

\begin{proof}
By induction on the structure of the derivation of $\Gamma \vdash A
\istrue$.

In class, we cover the $\imp E$ case:

  \[
  \DD = \infer[\imp E]{\Gamma \vdash B\istrue}
    {\deduce{\Gamma \vdash A \imp B\istrue}{\DD'} 
    & \deduce{\Gamma \vdash A\istrue}{\EE}}
  \]

  Need to show: $\Gamma \proves B$.

  By IH on $\DD'$, we have $\DD^* : \Gamma \proves A \imp B$.
  By IH on $\EE$, and then weakening,
  we have $\EE^* : \Gamma, A \imp B \proves A$.

  In sequent calculus, construct the following derivation
  $\FF$:
  \[
    \infer[\imp L]
    {\Gamma, A \imp B \proves B}
    {\deduce{\Gamma, A \imp B \proves A}{\EE^*}
    &
    \infer[\mathsf{id}_B]
    {\Gamma, A \imp B, B \proves B}{}}
  \]

  By cut admissibility on $\DD^*$ and $\FF$,
  with cut formula $A \imp B$,
  we have $\Gamma \proves B$ as needed.

  The introduction rules from natural deduction
  all follow directly via their corresponding right rules
  in sequent calculus.

  The other elimination rules proceed
  similarly to the shown case, using cut on the principal
  formula of the rule to connect its inductively assumed proof 
  on the right to a derivation we can construct that uses it on the left.
\end{proof}

\begin{exercise}
  Come up with a proof term assignment for sequent calculus proofs.
  Re-express the cases of the proof above as translating
  natural deduction proof terms to sequent calculus proof terms.

  Try ``running'' this translation on a non-normal STLC program,
  such as $x:A \vdash \pi_1 ((\lambda{y}.y)\;x, ())$.
  Document any observations or hypotheses you have about the
  results, and any other experiments you might want to run
  to test them.
\end{exercise}

\subsection{Remarks: Soundness of ND and Normalization for STLC}

Because the sequent calculus allows us to establish its consistency by easy
inspection (see previous lecture), the preceding development---cut
admissibility and the translation from natural deduction---combine to give
us the consistency of natural deduction. To see why, suppose there
exists a proof of $\cdot \vdash \bot \istrue$. Then by \ref{nd-sc}, there
must be a cut-free proof of the sequent $\cdot \proves \bot$, but we
previously ruled this out, so the premise cannot hold.

Because these proofs were both constructive, this also means we have a 
{\em procedure} for going from a natural deduction proof to some kind of
normal
form. However, that normal form is still expressed as a sequent calculus
derivation. To make the ``round trip'' back to natural deduction,
we need an intermediate calculus that expresses its normal forms.

\section{Verifications and Uses}

We now introduce a calculus that will serve as a waypoint between
natural deduction and sequent calculus, known variously as {\em
verifications and uses} (Pfenning), {\em intercalation} (Gentzen(?)),
{\em normal natural deduction} (citation needed), and {\em normal}
or {\em canonical lambda calculus} (TODO cite).

Judgments: $A \verif$ ($A$ has a verification); $A \use$ ($A$ may be used)

\[
  \infer[\iand I]
  {A \iand B \verif}
  {A \verif & B \verif}
  \qquad
  \infer[\iand E_1]
  {A \use}
  {A \iand B \use}
  \qquad
  \infer[\iand E_2]
  {B \use}
  {A \iand B \use}
  \qquad
  \infer[\top I]
  {\top \verif}{}
\]

\[
  \infer[\ior I_1]
  {A \ior B \verif}
  {A \verif}
  \qquad
  \infer[\ior I_2]
  {A \ior B \verif}
  {B \verif}
  \qquad
  \infer[\ior E]
  {C \verif}
  {A \ior B \use
   &
   A \use \vdash C \verif
   &
   B \use \vdash C \verif
  }
\]

\[
  \infer[\bot E]
  {C \verif}
  {\bot \use}
  \qquad
  \infer[\imp I]
  {A \imp B \verif}
  {A \use \vdash B \verif}
  \qquad
  \infer[\imp E]
  {B \use}
  {A \imp B \use
   &
   A \verif}
  \qquad
  \infer[\verif\use]
  {P \verif}
  {P \use}
\]

Note that we restrict switching from uses to verifications
(the $\verif \use$ rule) to atomic propositions $P$.

\subsection{Proof Terms}

At the end of lecture, we introduced a proof term language.
Because we have two judgments, we have two syntactic
categories, $R$ for {\em atomic} terms, i.e. proofs of $A\use$, 
and $M$ for {\em normal} terms, i.e. verifications. We argued (informally)
that all $M : A \verif$ are in $\beta$-short, $\eta$-long form,
and thus we cannot represent reducible expressions.
That is, all verifications are normal by construction.

\[
  \infer[\iand I]
  {(M_1, M_2) : A \iand B \verif}
  {M_1 : A \verif & M_2 : B \verif}
  \qquad
  \infer[\iand E_1]
  {\pi_1 R : A \use}
  {R : A \iand B \use}
  \qquad
  \infer[\iand E_2]
  {\pi_2 R : B \use}
  {R : A \iand B \use}
\]


\[
  \infer[\top I]
  {() : \top \verif}{}
  \qquad
  \infer[\ior I_1]
  {\exinl M : A \ior B \verif}
  {M : A \verif}
  \qquad
  \infer[\ior I_2]
  {\exinr M : A \ior B \verif}
  {M : B \verif}
\]
\[
  \infer[\ior E]
  {\excase{R, x.M, y.N} : C \verif}
  {R : A \ior B \use
   &
   x:A \use \vdash M:C \verif
   &
   y:B \use \vdash N:C \verif
  }
  \qquad
  \infer[\bot E]
  {\excase{R} : C \verif}
  {R : \bot \use}
\]
\[
  \infer[\imp I]
  {\lambda{x}.M : A \imp B \verif}
  {x:A \use \vdash M:B \verif}
  \qquad
  \infer[\imp E]
  {R\;M : B \use}
  {R : A \imp B \use
   &
   M : A \verif}
  \qquad
  \infer[\verif\use]
  {\{R\} : P \verif}
  {R : P \use}
\]


\bibliographystyle{plainnat}
\bibliography{main}

\end{document}
