\documentclass{article}
\usepackage{graphicx} % Required for inserting images
\usepackage{proof-dashed,amsmath,amssymb,amsthm}
\usepackage{xcolor}
\usepackage{stmaryrd}
\usepackage{microtype}
\usepackage[utf8]{inputenc}
\newcommand{\exif}[1]{\mathsf{if}\;{#1}}
\newcommand{\exthen}[1]{\ \mathsf{then}\;{#1}}
\newcommand{\exelse}[1]{\ \mathsf{else}\;{#1}}
\newcommand{\exite}[1]{\mathsf{ite}(#1)}
\newcommand{\explus}[1]{\mathsf{plus}(#1)}
\newcommand{\extrue}{\mathsf{true}}
\newcommand{\exfalse}{\mathsf{false}}
\newcommand{\exnum}[1]{\mathsf{num}(#1)}
\newcommand{\exbool}[1]{\mathsf{bool}(#1)}
\newcommand{\exlet}[1]{\mathsf{let}(#1)}
\newcommand{\exfun}[1]{\ensuremath{\lambda{#1}}}
% \newcommand{\exapp}[2]{\mathsf{app}({#1}, {#2})}
\newcommand{\exapp}[2]{{#1}{(#2)}}
\newcommand{\exunit}{()}
\newcommand{\expair}[1]{(#1)}
\newcommand{\exlpair}[1]{\langle #1 \rangle}
\newcommand{\exinl}[1]{\mathsf{in}_1\,{#1}}
\newcommand{\exinr}[1]{\mathsf{in}_2\,{#1}}
%\newcommand{\expil}[1]{\ensuremath{\pi_1}\,{#1}}
%\newcommand{\expir}[1]{\ensuremath{\pi_2}\,{#1}}
% "dot" syntax:
\newcommand{\expil}[1]{{#1}.1}
\newcommand{\expir}[1]{{#1}.2}
\newcommand{\excase}[1]{\mathsf{case}(#1)}
\newcommand{\exsplit}[1]{\mathsf{split}(#1)}
\newcommand{\exlam}[1]{\exfun{#1}}
\newcommand{\exzero}{\mathsf{zero}}
\newcommand{\exsucc}[1]{\mathsf{succ}(#1)}
\newcommand{\exrec}[1]{\mathsf{rec}(#1)}
\newcommand{\exdiverge}{\bot}
\newcommand{\exfix}[2]{\mathsf{fix}_{#1}(#2)}
\newcommand{\exifz}[1]{\mathsf{ifz}(#1)}
\newcommand{\extyab}[1]{\Lambda{#1}}
\newcommand{\extyapp}[2]{#1[#2]}
\newcommand{\exfail}{\mathsf{fail}}
\newcommand{\extry}[1]{\mathsf{trycatch}(#1)}
\newcommand{\exalloc}[1]{\mathsf{alloc}(#1)}
\newcommand{\exassign}[2]{{#1} := {#2}}
\newcommand{\exderef}[1]{?{#1}}

% types
\newcommand{\tybool}{\mathsf{Bool}}
\newcommand{\tynum}{\mathsf{Num}}
\newcommand{\tynat}{\mathsf{nat}}
\newcommand{\tyzero}{\mathbf{0}}
% \newcommand{\tyone}{\mathbb{1}}
\newcommand{\tyone}{\mathbf{1}}
% \newcommand{\tytwo}{\mathbb{2}}
\newcommand{\tytwo}{\mathbf{2}}
\newcommand{\typrod}{\times}
\newcommand{\tyfun}{\to}
\newcommand{\tysum}{+}
\newcommand{\tywith}{\;\&\;}
\newcommand{\typarfun}{\rightharpoonup}
\newcommand{\tyall}{\forall}
\newcommand{\tyref}[1]{\mathsf{ref}(#1)}
\newcommand{\tylolli}{\multimap}
\newcommand{\tyoplus}{\oplus}
\newcommand{\tytens}{\otimes}
\newcommand{\tytop}{\top}
\newcommand{\tybang}{\,!}

\newcommand{\pzero}{\mathsf{zero}}
\newcommand{\psucc}{\mathsf{succ}}

\newcommand{\evals}{\Downarrow}
\newcommand{\isval}{\ \mathsf{value}}
\newcommand{\istype}{\ \mathsf{type}}
\newcommand{\isctx}{\ \mathsf{ctx}}

\newcommand{\steps}{\mapsto}
\newcommand{\stepstar}{\mapsto^{*}}
\newcommand{\reduces}{\longrightarrow}

% Stacks
\newcommand{\seval}{\triangleright}
\newcommand{\sret}{\triangleleft}
\newcommand{\hole}{\square}
%\DeclareUnicodeCharacter{1F635}{\failstate}
%\DeclareRobustCommand\failstate{%
%  \unskip\nobreak\thinspace\textemdash\allowbreak\thinspace\ignorespaces}
\newcommand{\failstate}{\blacktriangleleft}

%% Program equivalence
\newcommand{\iso}{\simeq}
% \newcommand{\eequiv}{\cong}
\newcommand{\eequiv}{\approx}
\newcommand{\vequiv}{\sim}

% Semantics
\newcommand{\mathnat}{\mathbf{N}}
\newcommand{\encode}[1]{\ulcorner #1\urcorner}
\newcommand{\vsem}[1]{\llbracket #1 \rrbracket}
\newcommand{\rel}[1]{\mathbf{Rel}(#1)}

% Symbol reference: https://milde.users.sourceforge.net/LUCR/Math/mathpackages/amssymb-symbols.pdf

% Ordered logic
\newcommand{\ordMult}{\bullet}
\newcommand{\ordUnit}{\epsilon}
\newcommand{\ordArrL}{\rightarrowtail}
\newcommand{\ordArrR}{\twoheadrightarrow}
\newcommand{\ordCtx}{\Omega}
\newcommand{\gnab}{\text{\textexclamdown}}

% Linear logic
\newcommand{\linMult}{\otimes}
\newcommand{\linUnit}{\mathbb{1}}
\newcommand{\linArr}{\multimap}
\newcommand{\linWith}{\&}
\newcommand{\linTop}{\top}
\newcommand{\linPlus}{\oplus}
\newcommand{\linOne}{\mathbb{1}}
\newcommand{\linZero}{\mathbb{0}}
\newcommand{\linCtx}{\Delta}
\newcommand{\bang}{!}

% Intuitionistic logic
\newcommand{\iand}{\wedge}
\newcommand{\ior}{\vee}
\newcommand{\imp}{\supset}
\newcommand{\itrue}{\top}
\newcommand{\ifalse}{\bot}

% Shifts
\newcommand{\ushift}[2]{\uparrow\smash{{}^{#2}_{#1}}}
\newcommand{\dshift}[2]{\downarrow\smash{{}^{#1}_{#2}}}

% Judgments
\newcommand{\istrue}{\ \mathsf{true}}
\newcommand{\proves}{\Rightarrow}
\newcommand{\entails}{\vdash}
\newcommand{\hyp}{\ \mathsf{hyp}}
\newcommand{\conc}{\ \mathsf{conc}}
\newcommand{\verif}{\textcolor{blue}{\,\uparrow}}
\newcommand{\use}{\textcolor{red}{\,\downarrow}}

\newcommand{\DD}{\mathcal{D}}
\newcommand{\EE}{\mathcal{E}}
\newcommand{\FF}{\mathcal{F}}


\usepackage{aliascnt}

\newtheorem{theorem}{Theorem}

\newaliascnt{conjecture}{theorem}
\newtheorem{conjecture}[conjecture]{Conjecture}
\aliascntresetthe{conjecture}
\providecommand*{\conjectureautorefname}{Conjecture}
% \newtheorem{conjecture}[theorem]{Conjecture}

\newaliascnt{lemma}{theorem}
\newtheorem{lemma}[lemma]{Lemma}
\aliascntresetthe{lemma}
\providecommand*{\lemmaautorefname}{Lemma}
% \newtheorem{lemma}[theorem]{Lemma}

\newaliascnt{corollary}{theorem}
\newtheorem{corollary}[corollary]{Corollary}
\aliascntresetthe{corollary}
\providecommand*{\corollaryautorefname}{Corollary}
% \newtheorem{corollary}[theorem]{Corollary}

\newtheorem{exercise}{Exercise}
\providecommand*{\exerciseautorefname}{Exercise}

\newtheorem{discuss}{Discussion Question}
\providecommand*{\exerciseautorefname}{Discussion Question}

\newtheorem{definition}[theorem]{Definition}

% \newenvironment{proof}{\trivlist \item[\hskip \labelsep{\bf 
% Proof:}]}{\hfill$\Box$ \endtrivlist}
\newenvironment{sketch}{\trivlist \item[\hskip \labelsep{\bf 
Proof sketch:}]}{\hfill$\Box$ \endtrivlist}
\newenvironment{attempt}{\trivlist \item[\hskip \labelsep{\bf 
Proof attempt:}]}{\hfill$\Diamond$ \endtrivlist}

% Borrowed from Joseph's lec06.
\newcommand*{\bnfor}{\mid}
\newcommand*{\focsep}{\,\vert\,}
\newcommand*{\foc}[1]{\mathcolor{red}{#1}}
\newcommand*{\nat}{\operatorname{\textsf{nat}}}
\newcommand*{\s}{\operatorname{\textsf{suc}}}
\newcommand*{\z}{\operatorname{\textsf{zero}}}

\title{Lecture 7: Logic Programming}
\author{Mingtong Lin}
\date{September 29, 2025}

\begin{document}

\maketitle

\section{Introduction}

Lecture outline:
\begin{itemize}
\item Quantifiers
\item Logic programming
\end{itemize}

\section{Quantifiers}

\subsection{Grammar}

In this lecture, we continue with the polarized sequent calculus from last time,
but extended it with quantifiers.  The grammar and the blurring rules are the same
from last time, but the focusing rules are presented slightly differently.

\begin{alignat*}{2}
  &A, B ::= A^{+} \bnfor A^{-} &&\text{formulas} \\
  &A^{+}, B^{+} ::= \exists x : \tau. \, A &&\text{positive connectives} \\
  &A^{-}, B^{-} ::= A \imp B \bnfor \forall x : \tau. \, A \bnfor p \quad &&\text{negative connectives}
\end{alignat*}

\[
  \infer[\textsf{focus}^{u}_{L}]
  {\Gamma \proves C}
  {u : A^{-} \in \Gamma
    &
  \Gamma \focsep \foc{[A^{-}]} \proves C}
  \qquad
  \infer[\textsf{focus}_{R}]
  {\Gamma \proves A^{+}}
  {\Gamma \proves \foc{[A^{+}]}}
\]
\[
  \infer[\textsf{blur}_{L}]
  {\Gamma \focsep \foc{[A^{+}]} \proves C}
  {\Gamma, A^{+} \proves C}
  \qquad
  \infer[\textsf{blur}_{R}]
  {\Gamma \proves \foc{[A^{-}]}}
  {\Gamma \proves A^{-}}
\]
\[
  \infer[\textsf{init}^{-}]
  {\Gamma \focsep \foc{[p^{-}]} \proves p^{-}}
  {}
\]

\subsection{Logical Rules}

We first present the quantifier rules with some redundancies.
\[
  \infer[\forall R]
  {\Gamma \proves \forall x : \tau. \, A}
  {\Gamma, y : \tau \proves A [y/x]}
  \qquad
  \infer[\forall L]
  {\Gamma \focsep \foc{[\forall x : \tau. \, A]} \proves C}
  {\Gamma \proves t : \tau
    &
    \Gamma, \forall x : \tau. \, A \focsep \foc{[A[t/x]]} \proves C}
\]
\[
  \infer[\exists R]
  {\Gamma \proves \foc{[\exists x : \tau. \, A]}}
  {\Gamma \proves t : \tau
    &
    \Gamma \proves \foc{[A[t/x]]}}
  \qquad
  \infer[\exists L]
  {\Gamma, \exists x : \tau. \, A \proves C}
  {\Gamma, \exists x : \tau. \, A, t : \tau, u : A [t/x] \proves C}
\]
\[
  \infer[\imp R]
  {\Gamma \proves A \imp B}
  {\Gamma, u : A \proves B}
  \qquad
  \infer[\imp L]
  {\Gamma \focsep \foc{[A \imp B]} \proves C}
  {\Gamma, A \imp B \proves \foc{[A]}
    &
    \Gamma, A \imp B \focsep \foc{[B]} \proves C}
\]

The optimized rules are:
\[
  \infer[\forall R]
  {\Gamma \proves \forall x : \tau. \, A}
  {\Gamma, x : \tau \proves A}
  \qquad
  \infer[\forall L]
  {\Gamma \focsep \foc{[\forall x : \tau. \, A]} \proves C}
  {\Gamma \proves t : \tau
    &
    \Gamma \focsep \foc{[A[t/x]]} \proves C}
\]
\[
  \infer[\exists R]
  {\Gamma \proves \foc{[\exists x : \tau. \, A]}}
  {\Gamma \proves t : \tau
    &
    \Gamma \proves \foc{[A[t/x]]}}
  \qquad
  \infer[\exists L]
  {\Gamma, \exists x : \tau. \, A \proves C}
  {\Gamma, \exists x : \tau. \, A, t : \tau, u : A [t/x] \proves C}
\]
\[
  \infer[\imp R]
  {\Gamma \proves A \imp B}
  {\Gamma, u : A \proves B}
  \qquad
  \infer[\imp L]
  {\Gamma \focsep \foc{[A \imp B]} \proves C}
  {\Gamma \proves \foc{[A]}
    &
    \Gamma \focsep \foc{[B]} \proves C}
\]

\section{Logic Programming}

For the rest of the lecture, we remove \(\exists R\) and \(\exists L\).  Since existential
quantifier is the only positive connective in our grammar, we have essentially removed the
entire syntax category \(A^{+}\).  Also, since we no longer have any positive rules, we may remove
\(\textsf{blur}_{L}\) and \(\textsf{focus}_{R}\) as well.  Then, we are left with the negative
fragment of the logic.

For the following example:
\[
  \cdot \proves p \imp (p \imp q) \imp (q \imp r) \imp (p \imp r) \imp r
\]
First, we apply the trivial \(\imp R\) rules:
\[
  \infer[\imp R]
  {\cdot \proves p \imp (p \imp q) \imp (q \imp r) \imp (p \imp r) \imp r}
  {\infer[\imp R]
    {u_{1} : p \proves (p \imp q) \imp (q \imp r) \imp (p \imp r) \imp r}
    {\infer[\imp R]
      {u_{1} : p, u_{2} : p \imp q \proves (q \imp r) \imp (p \imp r) \imp r}
      {\infer[\imp R]
        {u_{1} : p, u_{2} : p \imp q, u_{3} : q \imp r \proves (p \imp r) \imp r}
        {\infer[\imp R]
        {\Gamma = u_{1} : p, u_{2} : p \imp q, u_{3} : q \imp r, u_{4} : p \imp r \proves r}
        {?}}}}}
\]
Let's pause and rewrite the sequent as \(\Gamma \proves C\).  Now, think about what are
the possible rules to apply.  Clearly, the only rule that gets us foward is
\(\textsf{focus}^{u}_{L}\), but which \(u\) should we choose?

Let's first try \(u_{1} : p\), then we end up with the derivation:
\[
  \infer[\textsf{focus}^{u_{1}}_{L}]
  {\Gamma \proves C}
  {\infer[?]
    {\Gamma \focsep \foc{[p]} \proves C}
    {}}
\]
The only choice for \(?\) seems to be \(\textsf{init}^{-}\):
\[
  \infer[\textsf{focus}^{u_{1}}_{L}]
  {\Gamma \proves C}
  {\infer[\textsf{init}^{-}]
    {\Gamma \focsep \foc{[p]} \proves p^{-}}
    {}}
\]
But then, this means that our goal \(C\) must be \(p\), which is not the case.
We may also distill the ``knowledge'' we just learned from this trial into a
\textit{synthetic inference rule}:
\[
  \infer[]
  {\Gamma \proves C}
  {C = p^{-}}
\]

Similarly, if we choose to focus on \(u_{3}\), we get:
\[
  \infer[\textsf{focus}^{u_{3}}_{L}]
  {\Gamma \proves C}
  {\infer[\imp L]
    {\Gamma \focsep \foc{[q \imp r]} \proves C}
    {\infer[\textsf{blur}_{R}]
      {\Gamma \proves \foc{[q]}}
      {\Gamma \proves q}
      &
      \infer[\textsf{init}^{-}]
      {\Gamma \focsep \foc{[r]} \proves r}
      {}}}
\]
and the synthetic inference rule:
\[
  \infer[]
  {\Gamma \proves r}
  {\Gamma \proves q}
\]
This looks plausible, as our goal is indeed \(r\)!  Continue with the proof, the full
derivation turns out to be:
\[
  \infer[\textsf{focus}^{u_{3}}_{L}]
  {\Gamma \proves r}
  {\infer[\imp L]
    {\Gamma \focsep \foc{[q \imp r]} \proves r}
    {\infer[\textsf{blur}_{R}]
      {\Gamma \proves \foc{[q]}}
      {\infer[\textsf{focus}^{u_{2}}_{L}]
        {\Gamma \proves q}
        {\infer[\imp L]
          {\Gamma \focsep \foc{[p \imp q]} \proves q}
          {\infer[\textsf{blur}_{R}]
            {\Gamma \proves \foc{[p]}}
            {\infer[\textsf{focus}^{u_{1}}_{L}]
              {\Gamma \proves p}
              {\infer[\textsf{init}^{-}]
                {\Gamma \focsep \foc{[p]} \proves p}
                {}}}
            &
            \infer[\textsf{init}^{-}]
            {\Gamma \focsep \foc{[q]} \proves q}
            {}}}}
      &
      \infer[\textsf{init}^{-}]
      {\Gamma \focsep \foc{[r]} \proves r}
      {}}}
\]

Intuitively, if we focus on a formula of form \(p \imp \cdots \imp s\) for proving
\(\Gamma \proves C\), the goal \(C\) must be exactly \(s\).

\section{Logic Programming in Twelf}

\subsection{A Quick Taste of Twelf}

Let's try putting the sequent we proved above into Twelf:
\begin{verbatim}
p : type.
q : type.
r : type.

%solve mypf :
p -> (p -> q) -> (q -> r) -> (p -> r) -> r.

%query 2 *
p -> (p -> q) -> (q -> r) -> (p -> r) -> r.
\end{verbatim}

Here, we try first to \texttt{\%solve} the sequent, which will search for an inhabitant of it.
Then, we use the \texttt{\%query} directive to ask Twelf to search for two derivations of the
proof, with no depth limit.  It turns out that we could either focus on \(p \imp r\) or
\(q \imp r\), and they both get us the proof.

Alternatively, we can name the premises as axioms and directly solve for \(r\):
\begin{verbatim}
trueP : p.
pImpQ : p -> q.
qImpR : q -> r.

%solve mypf: r.
\end{verbatim}
This will give us a more readable derivation with the names in place of the formulae.

\subsection{Proof Search for Natural Numbers}

Twelf allows us to define custom inductive types.  Here's a good example of how logic
programming languages like Twelf generalize the above recipe to work with them.

\begin{verbatim}
nat : type.
z : nat.
s : nat -> nat.

lt : nat -> nat -> type.
lt/z : lt z (s N).
lt/s : lt N M -> lt (s N) (s M).

%solve 2<3 : lt (s (s z)) (s (s (s z))).
\end{verbatim}

The \texttt{2<3} in the above Twelf program can be translated to
\[
  \Gamma \proves a < b
\]
where \(a = \s \s \z\) and \(b = \s \s \s \z\), and our context \(\Gamma\) contains the
equivalent definitions of \texttt{lt}:
\begin{align*}
  \mathtt{lt/z} &: \forall n : \nat. \, \z < \s n \\
  \mathtt{lt/s} &: \forall n : \nat. \, \forall m : \nat. \, (n < m) \imp (\s n < \s m)
\end{align*}

As usual, let's first think about the possible derivation of \(\Gamma \proves C\).
Apparently, the goal hints that we need  to apply one of the \texttt{lt} rules.  Then,
we recognize that the \texttt{lt/z} rule does not apply since neither of \(a\) or \(b\)
is \(\z\).  Therefore, we can only choose to focus on the \texttt{lt/s} rule, and we get
the derivation:
\[
  \infer[\textsf{focus}_{L}]
  {\Gamma \proves C}
  {\infer[\forall L]
    {\Gamma \focsep \foc{[\forall n : \nat. \, \forall m : \nat. \, (n < m) \imp (\s n < \s m)]} \proves C}
    {\Gamma \proves a : \nat
      &
      \infer[\forall L]
      {\Gamma \focsep \foc{[\forall m : \nat. \, (a < m) \imp (\s a < \s m)]} \proves C}
      {\Gamma \proves b : \nat
        &
        \infer[\imp L]
        {\Gamma \focsep \foc{[(a < b) \imp (\s a < \s b)]} \proves C}
        {\infer[\textsf{blur}_{R}]
          {\Gamma \proves \foc{[a < b]}}
          {\Gamma \proves a < b}
          &
          \Gamma \focsep \foc{[\s a < \s b]} \proves C}}}}
\]

\end{document}
