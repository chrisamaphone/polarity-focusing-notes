\documentclass{article}
\usepackage{graphicx} % Required for inserting images
\usepackage{proof-dashed,amsmath,amssymb,amsthm}
\usepackage{xcolor}
\usepackage{stmaryrd}
\usepackage{microtype}
\usepackage[utf8]{inputenc}
\usepackage{bbold}
\usepackage{mathtools}
\input{program-macros}
\input{logic-macros}
\input{metatheory}

\title{Assignment 2: Focusing and Polarity}
\author{Chris Martens}
\date{\today}

\begin{document}

\maketitle

This assignment covers exercises from lectures 6 through 9.

\section{Lecture 6 Exercises}
\begin{exercise}
Show that the implication right rule is invertible and that the disjunction right rules are not.
\end{exercise}

\begin{exercise}
Think about what it would look like to have a connective that is \emph{neither} positive nor
negative. Why would this be interesting? Try to design such a connective, and discuss your findings.
\end{exercise}


\begin{exercise}
  Write the rules for (intuitionistic) conjunction treated as a positive connective.
\end{exercise}


\begin{exercise}
  Give a focused proof of the following sequent:
  \[ (p \imp q) \ior (p \imp r) \proves p \imp (q \ior r) \]
  
  Is there more than one possible proof? If so, explain why.
\end{exercise}


\begin{exercise}
Show that the universal quantifier is negative.
\end{exercise}


\section{Lecture 7 Exercises}

(None given)


\section{Lecture 8 Exercises}


\begin{exercise}
  Prove the linear logic sequent 
  \[\cdot \proves A \lolli B \lolli C \proves A \tensor B \lolli C\]
\end{exercise}

The following were not given as exercises in class, but can be helpful
for testing your understanding of linear logic.

\newcommand{\eqhuh}{\xLeftrightarrow{\text{?}}}

\begin{exercise}
  For each of the below schema, state whether one can be proven from the
  other,
  both can be proven from each other, or neither.
  Check your work with proofs (but you do not need to include your proofs
  in your hand-in).
  \begin{enumerate}
    \item $A \oplus A \eqhuh (\one \oplus \one) \tensor A$
    \item $A \tensor (B \oplus C) \eqhuh (A \tensor B) \oplus (A \tensor
      C$
    \item $A \with B \eqhuh A \oplus B$
    \item $A \with (B \oplus C) \eqhuh (A \with B) \oplus (A \with C)$
    \item $(A \with B) \oplus (A \with C) \eqhuh A \with (B \oplus C)$
    \item $A \oplus B \lolli C \eqhuh (A \lolli C) \with (B \lolli C)$
  \end{enumerate}
\end{exercise}



\section{Lecture 9 Exercises}

\begin{exercise}
 Both $\tensor R$ and $\with L$ are non-invertible rules.
  Give another proof of the same proposition
  ($\cdot \proves (p \with q) \lolli (r \with s) \lolli (p \tensor r)$)
  where they are used in the other order.
\end{exercise}


\begin{exercise}
Translate your proof of
  \[
    \cdot \proves (p \with q) \lolli (r \with s) \lolli (p \tensor r)
  \]
from the previous exercise, choosing an appropriate polarization
for the atomic propositions. Explain how the structure of the proof
forces such a polarization.
\end{exercise}



\end{document}
