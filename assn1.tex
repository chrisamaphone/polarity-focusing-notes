\documentclass{article}
\usepackage{graphicx} % Required for inserting images
\usepackage{proof-dashed,amsmath,amssymb,amsthm}
\usepackage{xcolor}
\usepackage{stmaryrd}
\usepackage{microtype}
\usepackage[utf8]{inputenc}
\newcommand{\exif}[1]{\mathsf{if}\;{#1}}
\newcommand{\exthen}[1]{\ \mathsf{then}\;{#1}}
\newcommand{\exelse}[1]{\ \mathsf{else}\;{#1}}
\newcommand{\exite}[1]{\mathsf{ite}(#1)}
\newcommand{\explus}[1]{\mathsf{plus}(#1)}
\newcommand{\extrue}{\mathsf{true}}
\newcommand{\exfalse}{\mathsf{false}}
\newcommand{\exnum}[1]{\mathsf{num}(#1)}
\newcommand{\exbool}[1]{\mathsf{bool}(#1)}
\newcommand{\exlet}[1]{\mathsf{let}(#1)}
\newcommand{\exfun}[1]{\ensuremath{\lambda{#1}}}
% \newcommand{\exapp}[2]{\mathsf{app}({#1}, {#2})}
\newcommand{\exapp}[2]{{#1}{(#2)}}
\newcommand{\exunit}{()}
\newcommand{\expair}[1]{(#1)}
\newcommand{\exlpair}[1]{\langle #1 \rangle}
\newcommand{\exinl}[1]{\mathsf{in}_1\,{#1}}
\newcommand{\exinr}[1]{\mathsf{in}_2\,{#1}}
%\newcommand{\expil}[1]{\ensuremath{\pi_1}\,{#1}}
%\newcommand{\expir}[1]{\ensuremath{\pi_2}\,{#1}}
% "dot" syntax:
\newcommand{\expil}[1]{{#1}.1}
\newcommand{\expir}[1]{{#1}.2}
\newcommand{\excase}[1]{\mathsf{case}(#1)}
\newcommand{\exsplit}[1]{\mathsf{split}(#1)}
\newcommand{\exlam}[1]{\exfun{#1}}
\newcommand{\exzero}{\mathsf{zero}}
\newcommand{\exsucc}[1]{\mathsf{succ}(#1)}
\newcommand{\exrec}[1]{\mathsf{rec}(#1)}
\newcommand{\exdiverge}{\bot}
\newcommand{\exfix}[2]{\mathsf{fix}_{#1}(#2)}
\newcommand{\exifz}[1]{\mathsf{ifz}(#1)}
\newcommand{\extyab}[1]{\Lambda{#1}}
\newcommand{\extyapp}[2]{#1[#2]}
\newcommand{\exfail}{\mathsf{fail}}
\newcommand{\extry}[1]{\mathsf{trycatch}(#1)}
\newcommand{\exalloc}[1]{\mathsf{alloc}(#1)}
\newcommand{\exassign}[2]{{#1} := {#2}}
\newcommand{\exderef}[1]{?{#1}}

% types
\newcommand{\tybool}{\mathsf{Bool}}
\newcommand{\tynum}{\mathsf{Num}}
\newcommand{\tynat}{\mathsf{nat}}
\newcommand{\tyzero}{\mathbf{0}}
% \newcommand{\tyone}{\mathbb{1}}
\newcommand{\tyone}{\mathbf{1}}
% \newcommand{\tytwo}{\mathbb{2}}
\newcommand{\tytwo}{\mathbf{2}}
\newcommand{\typrod}{\times}
\newcommand{\tyfun}{\to}
\newcommand{\tysum}{+}
\newcommand{\tywith}{\;\&\;}
\newcommand{\typarfun}{\rightharpoonup}
\newcommand{\tyall}{\forall}
\newcommand{\tyref}[1]{\mathsf{ref}(#1)}
\newcommand{\tylolli}{\multimap}
\newcommand{\tyoplus}{\oplus}
\newcommand{\tytens}{\otimes}
\newcommand{\tytop}{\top}
\newcommand{\tybang}{\,!}

\newcommand{\pzero}{\mathsf{zero}}
\newcommand{\psucc}{\mathsf{succ}}

\newcommand{\evals}{\Downarrow}
\newcommand{\isval}{\ \mathsf{value}}
\newcommand{\istype}{\ \mathsf{type}}
\newcommand{\isctx}{\ \mathsf{ctx}}

\newcommand{\steps}{\mapsto}
\newcommand{\stepstar}{\mapsto^{*}}
\newcommand{\reduces}{\longrightarrow}

% Stacks
\newcommand{\seval}{\triangleright}
\newcommand{\sret}{\triangleleft}
\newcommand{\hole}{\square}
%\DeclareUnicodeCharacter{1F635}{\failstate}
%\DeclareRobustCommand\failstate{%
%  \unskip\nobreak\thinspace\textemdash\allowbreak\thinspace\ignorespaces}
\newcommand{\failstate}{\blacktriangleleft}

%% Program equivalence
\newcommand{\iso}{\simeq}
% \newcommand{\eequiv}{\cong}
\newcommand{\eequiv}{\approx}
\newcommand{\vequiv}{\sim}

% Semantics
\newcommand{\mathnat}{\mathbf{N}}
\newcommand{\encode}[1]{\ulcorner #1\urcorner}
\newcommand{\vsem}[1]{\llbracket #1 \rrbracket}
\newcommand{\rel}[1]{\mathbf{Rel}(#1)}

% Symbol reference: https://milde.users.sourceforge.net/LUCR/Math/mathpackages/amssymb-symbols.pdf

% Ordered logic
\newcommand{\ordMult}{\bullet}
\newcommand{\ordUnit}{\epsilon}
\newcommand{\ordArrL}{\rightarrowtail}
\newcommand{\ordArrR}{\twoheadrightarrow}
\newcommand{\ordCtx}{\Omega}
\newcommand{\gnab}{\text{\textexclamdown}}

% Linear logic
\newcommand{\linMult}{\otimes}
\newcommand{\linUnit}{\mathbb{1}}
\newcommand{\linArr}{\multimap}
\newcommand{\linWith}{\&}
\newcommand{\linTop}{\top}
\newcommand{\linPlus}{\oplus}
\newcommand{\linOne}{\mathbb{1}}
\newcommand{\linZero}{\mathbb{0}}
\newcommand{\linCtx}{\Delta}
\newcommand{\bang}{!}

% Intuitionistic logic
\newcommand{\iand}{\wedge}
\newcommand{\ior}{\vee}
\newcommand{\imp}{\supset}
\newcommand{\itrue}{\top}
\newcommand{\ifalse}{\bot}

% Shifts
\newcommand{\ushift}[2]{\uparrow\smash{{}^{#2}_{#1}}}
\newcommand{\dshift}[2]{\downarrow\smash{{}^{#1}_{#2}}}

% Judgments
\newcommand{\istrue}{\ \mathsf{true}}
\newcommand{\proves}{\Rightarrow}
\newcommand{\entails}{\vdash}
\newcommand{\hyp}{\ \mathsf{hyp}}
\newcommand{\conc}{\ \mathsf{conc}}
\newcommand{\verif}{\textcolor{blue}{\,\uparrow}}
\newcommand{\use}{\textcolor{red}{\,\downarrow}}

\newcommand{\DD}{\mathcal{D}}
\newcommand{\EE}{\mathcal{E}}
\newcommand{\FF}{\mathcal{F}}


\usepackage{aliascnt}

\newtheorem{theorem}{Theorem}

\newaliascnt{conjecture}{theorem}
\newtheorem{conjecture}[conjecture]{Conjecture}
\aliascntresetthe{conjecture}
\providecommand*{\conjectureautorefname}{Conjecture}
% \newtheorem{conjecture}[theorem]{Conjecture}

\newaliascnt{lemma}{theorem}
\newtheorem{lemma}[lemma]{Lemma}
\aliascntresetthe{lemma}
\providecommand*{\lemmaautorefname}{Lemma}
% \newtheorem{lemma}[theorem]{Lemma}

\newaliascnt{corollary}{theorem}
\newtheorem{corollary}[corollary]{Corollary}
\aliascntresetthe{corollary}
\providecommand*{\corollaryautorefname}{Corollary}
% \newtheorem{corollary}[theorem]{Corollary}

\newtheorem{exercise}{Exercise}
\providecommand*{\exerciseautorefname}{Exercise}

\newtheorem{discuss}{Discussion Question}
\providecommand*{\exerciseautorefname}{Discussion Question}

\newtheorem{definition}[theorem]{Definition}

% \newenvironment{proof}{\trivlist \item[\hskip \labelsep{\bf 
% Proof:}]}{\hfill$\Box$ \endtrivlist}
\newenvironment{sketch}{\trivlist \item[\hskip \labelsep{\bf 
Proof sketch:}]}{\hfill$\Box$ \endtrivlist}
\newenvironment{attempt}{\trivlist \item[\hskip \labelsep{\bf 
Proof attempt:}]}{\hfill$\Diamond$ \endtrivlist}

\title{Assignment 1: Natural Deduction and Sequent Calculus}
\author{Chris Martens}
\date{\today}

\begin{document}

\maketitle

\section{Lecture 1 Exercises}

\subsection{Natural Deduction}
\begin{itemize}
  \item $(A \imp B) \iand A \imp B$
  \item $(A \imp (B \ior C)) \imp (A \iand \neg B) \imp C$
  \item $(A \ior B) \iand C \imp (A \iand C) \ior (B \iand C)$
\end{itemize}

\begin{exercise}\label{ex:derivs}
  Typeset derivations for these proofs.
\end{exercise}

\begin{exercise}
  \label{ex:hypnotation}
  In class, there was a discussion of the meaning of
  $\vdash$, the notation for the hypothetical judgment.
  Recall that we give the hypothetical judgment
  $B_1 \istrue, \ldots, B_n \istrue \vdash A \istrue$
  its meaning via ``scoped axioms'' of the form
  \[
    \infer[]{B_i \istrue}{}
  \]
  in the ambient metalogic, which are permitted to be
  used when proving $A \istrue$.

  In Gentzen's original notation, he elided the explicit mention of
  $\Gamma$, instead writing
  \[
  \infer[\imp I]
  {A \imp B \istrue}
  {\deduce{B \istrue}{[A \istrue]}}
  \]

  When we write $\Gamma$ explicitly, it is also very tempting
  to treat it as ``mere'' syntax that can be manipulated by
  inference rules (e.g. Weakening, Contraction, and Exchange; or an
  equational theory on contexts).

  These presentational choices impact both metatheory and
  pedagogy. For this exercise, decide for yourself on an approach that you
  might try using to present natural deduction to a audience unfamiliar
  with it: would you include $\Gamma$ explicitly or use Gentzen's (or a
  similar) notation? How would you write the hypothetical rule?
  Justify your choices.
  (If you like the presentation from lecture/these notes, that's fine, 
  but you still have to argue for it.)
  
\end{exercise}

\subsection{Counting Proofs}

There are infinitely many proofs of
\[
  \cdot \vdash (A \imp B) \imp (A \iand B) \imp B \istrue
\]


\begin{exercise}
  Write two distinct proofs of 
  the above judgment.
\end{exercise}

\begin{exercise}
  Informally describe a procedure for generating infinitely many
  such proofs.
\end{exercise}

\subsection{Proof Terms}

\begin{exercise}
  Rewrite the derivations from Exercise \ref{ex:derivs} in proof term notation.
\end{exercise}

\begin{exercise}
  Revisit your decision from Exercise \ref{ex:hypnotation}.
  How do your choices affect the information recorded by the
  proof term and its relationship to the derivation?
  If necessary, design a modification to the proof term
  syntax to account for your decision.
\end{exercise}


\section{Lecture 2 Exercises}

\begin{exercise}
    Show that the rules for implication are locally sound and complete, and show how they relate to $\beta$-reduction and $\eta$-expansion.
\end{exercise}


\section{Lecture 3 Exercises}

\subsection{Sequent Calculus}
\begin{itemize}
  \item $A \ior B \imp A \ior B$
  \item $(A \imp B) \iand A \imp B$
  \item $(A \imp (B \ior C)) \imp (A \iand \neg B) \imp C$
  \item $(A \ior B) \iand C \imp (A \iand C) \ior (B \iand C)$
\end{itemize}

\begin{exercise}
  Typeset proofs of $\cdot \proves A$ (in sequent calculus)
  for each formula $A$ above.
\end{exercise}


\begin{exercise} 
  State and prove a corresponding theorem for Contraction.
  If you are unable to make the proof rigorous,
  discuss why and what might be done to address it.
\end{exercise}

\subsection{Non-provability}

In sequent calculus, it is easy to demonstrate
that certain sequents are {\em not} provable.
We will work through some examples:

\begin{itemize}
  \item Soundness: no proof of $\cdot \proves \bot$
  \item Disjunction property: if $\cdot \proves A \ior B$,
    then $\cdot \proves A$ or $\cdot \proves B$.
  \item Non-provability of excluded middle (LEM): there is no proof
    of $\cdot \proves A \ior \lnot A$ for arbitrary $A$.
  \item Non-provability of double-negation elimination (DNE):
    there is no proof of $\cdot \proves \lnot \lnot A \implies A$.
\end{itemize}

\begin{exercise}
  Typeset proofs of the above non-provability arguments.
\end{exercise}

\begin{exercise}
  Find another classically valid, but not intuitionistically valid,
  proposition $A$ and demonstrate that $\cdot \proves A$ is not provable
  in general.
\end{exercise}

\begin{exercise}
  Why do we keep qualifying each of these statements with ``in general''
  or ``for arbitrary $A$''? Explain how the situation changes
  when we are allowed to talk about specific propositions $A$.
\end{exercise}


\begin{exercise}
  Discuss why the Cut and Identity metatheorem statements correspond to soundness and
  completeness for sequent calculus.
\end{exercise}

\section{Lecture 4 Exercises}


\begin{exercise}
  Come up with a proof term assignment for sequent calculus proofs.
  Re-express the cases of the proof above as translating
  natural deduction proof terms to sequent calculus proof terms.

  Try ``running'' this translation on a non-normal STLC program,
  such as $x:A \vdash \pi_1 ((\lambda{y}.y)\;x, ())$.
  Document any observations or hypotheses you have about the
  results, and any other experiments you might want to run
  to test them.
\end{exercise}




\end{document}
