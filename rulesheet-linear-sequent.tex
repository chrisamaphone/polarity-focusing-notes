\documentclass{article}
\usepackage{graphicx} % Required for inserting images
\usepackage{proof-dashed,amsmath,amssymb,amsthm}
\usepackage{xcolor}
\usepackage{bbold}
\usepackage{stmaryrd}
\usepackage{microtype}
\usepackage[utf8]{inputenc}
\input{program-macros}
\input{logic-macros}
\input{metatheory}

\usepackage[T1]{fontenc}
\usepackage{palatino}

% \setlength{\voffset}{-0.75in}
\setlength{\headsep}{5pt}
% \setlength{\paperheight}{1.1\paperheight}

\title{Rule Sheet: Linear Sequent Calculus}
\author{Chris Martens}
\date{\today}

\begin{document}

\maketitle

\section{Inference Rules}

Tensor:
\[
  \infer[\tensor R]
  {\Delta_1, \Delta_2 \proves A \tensor B}
  {\Delta_1 \proves A
  &
  \Delta_2 \proves B}
\qquad
  \infer[\tensor L]
  {\Delta, A \tensor B \proves C}
  {\Delta, A, B \proves C}
\]

With:
\[
  \infer[\with R]
  {\Delta \proves A \with B}
  {\Delta \proves A
    &
   \Delta \proves B}
\qquad
  \infer[\with L_1]
  {\Delta, A \with B \proves C}
  {\Delta, A \proves C}
\qquad
  \infer[\with L_2]
  {\Delta, A \with B \proves C}
  {\Delta, B \proves C}
\]

Disjunction (oplus):
\[
  \infer[\oplus R_1]
  {\Delta \proves A \oplus B}
  {\Delta \proves A}
\qquad
  \infer[\oplus R_2]
  {\Delta \proves A \oplus B}
  {\Delta \proves B}
\qquad
  \infer[\oplus L]
  {\Delta, A \oplus B \proves C}
  {\Delta, A \proves C
   &
   \Delta, B \proves C
  }
\]

Implication (lolli):
\[
  \infer[\lolli R]
  {\Delta \proves A \lolli B}
  {\Delta, A \proves B}
\qquad
  \infer[\lolli L]
  {\Delta_1, \Delta_2, A \lolli B \proves C}
  {\Delta_1 \proves A
  &
  \Delta_2, B \proves C}
\]


Positive unit:
\[
  \infer[\one R]
  {\Delta \proves \one}
  {\Delta = \cdot}
  \qquad
  \infer[\one L]
  {\Delta, \one \proves C }
  {\Delta \proves C}
\]

Negative and positive zero:
\[
  \infer[\top R]
  {\Delta \proves \top}{}
  \qquad
  \mathrm{(no\ \top L)}
  \qquad
  \mathrm{(no\ \zero R)}
  \qquad
  \infer[\zero L]
  {\Delta, \zero \proves C} {}
\]


Identity rule:
\[
  \infer[\mathsf{id}]
  {A \proves A}
  {}
\]

\section{Syntax}

Judgment $\Delta \proves A$,
where $\Delta$ is an unordered list
of propositions $A$. We may also write $\Gamma$ as a
context metavariable when it is clear that this does
not represent an unrestricted context.

Proposition forms $A,B,C$ include:
\begin{itemize}
  \item $A \tensor B$ (positive conjunction)
  \item $A \with B$ (negative conjunction)
  \item $A \oplus B$ (positive disjunction)
  \item $\one$ (unit of $\tensor$/positive unit)
  \item $\zero$ (unit of $\oplus$/positive zero)
  \item $\top$ (unit of $\with$/negative zero)
  \item $p$ (atomic propositions/placeholders for arbitrary propositions)
\end{itemize}


\bibliographystyle{plainnat}
\bibliography{main}

\end{document}
