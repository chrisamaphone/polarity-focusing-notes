\documentclass{article}
\usepackage{graphicx} % Required for inserting images
\usepackage{proof-dashed,amsmath,amssymb,amsthm}
\usepackage{xcolor}
\usepackage{stmaryrd}
\usepackage{microtype}
\usepackage[utf8]{inputenc}
\usepackage{bbold}

\newcommand{\exif}[1]{\mathsf{if}\;{#1}}
\newcommand{\exthen}[1]{\ \mathsf{then}\;{#1}}
\newcommand{\exelse}[1]{\ \mathsf{else}\;{#1}}
\newcommand{\exite}[1]{\mathsf{ite}(#1)}
\newcommand{\explus}[1]{\mathsf{plus}(#1)}
\newcommand{\extrue}{\mathsf{true}}
\newcommand{\exfalse}{\mathsf{false}}
\newcommand{\exnum}[1]{\mathsf{num}(#1)}
\newcommand{\exbool}[1]{\mathsf{bool}(#1)}
\newcommand{\exlet}[1]{\mathsf{let}(#1)}
\newcommand{\exfun}[1]{\ensuremath{\lambda{#1}}}
% \newcommand{\exapp}[2]{\mathsf{app}({#1}, {#2})}
\newcommand{\exapp}[2]{{#1}{(#2)}}
\newcommand{\exunit}{()}
\newcommand{\expair}[1]{(#1)}
\newcommand{\exlpair}[1]{\langle #1 \rangle}
\newcommand{\exinl}[1]{\mathsf{in}_1\,{#1}}
\newcommand{\exinr}[1]{\mathsf{in}_2\,{#1}}
%\newcommand{\expil}[1]{\ensuremath{\pi_1}\,{#1}}
%\newcommand{\expir}[1]{\ensuremath{\pi_2}\,{#1}}
% "dot" syntax:
\newcommand{\expil}[1]{{#1}.1}
\newcommand{\expir}[1]{{#1}.2}
\newcommand{\excase}[1]{\mathsf{case}(#1)}
\newcommand{\exsplit}[1]{\mathsf{split}(#1)}
\newcommand{\exlam}[1]{\exfun{#1}}
\newcommand{\exzero}{\mathsf{zero}}
\newcommand{\exsucc}[1]{\mathsf{succ}(#1)}
\newcommand{\exrec}[1]{\mathsf{rec}(#1)}
\newcommand{\exdiverge}{\bot}
\newcommand{\exfix}[2]{\mathsf{fix}_{#1}(#2)}
\newcommand{\exifz}[1]{\mathsf{ifz}(#1)}
\newcommand{\extyab}[1]{\Lambda{#1}}
\newcommand{\extyapp}[2]{#1[#2]}
\newcommand{\exfail}{\mathsf{fail}}
\newcommand{\extry}[1]{\mathsf{trycatch}(#1)}
\newcommand{\exalloc}[1]{\mathsf{alloc}(#1)}
\newcommand{\exassign}[2]{{#1} := {#2}}
\newcommand{\exderef}[1]{?{#1}}

% types
\newcommand{\tybool}{\mathsf{Bool}}
\newcommand{\tynum}{\mathsf{Num}}
\newcommand{\tynat}{\mathsf{nat}}
\newcommand{\tyzero}{\mathbf{0}}
% \newcommand{\tyone}{\mathbb{1}}
\newcommand{\tyone}{\mathbf{1}}
% \newcommand{\tytwo}{\mathbb{2}}
\newcommand{\tytwo}{\mathbf{2}}
\newcommand{\typrod}{\times}
\newcommand{\tyfun}{\to}
\newcommand{\tysum}{+}
\newcommand{\tywith}{\;\&\;}
\newcommand{\typarfun}{\rightharpoonup}
\newcommand{\tyall}{\forall}
\newcommand{\tyref}[1]{\mathsf{ref}(#1)}
\newcommand{\tylolli}{\multimap}
\newcommand{\tyoplus}{\oplus}
\newcommand{\tytens}{\otimes}
\newcommand{\tytop}{\top}
\newcommand{\tybang}{\,!}

\newcommand{\pzero}{\mathsf{zero}}
\newcommand{\psucc}{\mathsf{succ}}

\newcommand{\evals}{\Downarrow}
\newcommand{\isval}{\ \mathsf{value}}
\newcommand{\istype}{\ \mathsf{type}}
\newcommand{\isctx}{\ \mathsf{ctx}}

\newcommand{\steps}{\mapsto}
\newcommand{\stepstar}{\mapsto^{*}}
\newcommand{\reduces}{\longrightarrow}

% Stacks
\newcommand{\seval}{\triangleright}
\newcommand{\sret}{\triangleleft}
\newcommand{\hole}{\square}
%\DeclareUnicodeCharacter{1F635}{\failstate}
%\DeclareRobustCommand\failstate{%
%  \unskip\nobreak\thinspace\textemdash\allowbreak\thinspace\ignorespaces}
\newcommand{\failstate}{\blacktriangleleft}

%% Program equivalence
\newcommand{\iso}{\simeq}
% \newcommand{\eequiv}{\cong}
\newcommand{\eequiv}{\approx}
\newcommand{\vequiv}{\sim}

% Semantics
\newcommand{\mathnat}{\mathbf{N}}
\newcommand{\encode}[1]{\ulcorner #1\urcorner}
\newcommand{\vsem}[1]{\llbracket #1 \rrbracket}
\newcommand{\rel}[1]{\mathbf{Rel}(#1)}

% Symbol reference: https://milde.users.sourceforge.net/LUCR/Math/mathpackages/amssymb-symbols.pdf

% Ordered logic
\newcommand{\ordMult}{\bullet}
\newcommand{\ordUnit}{\epsilon}
\newcommand{\ordArrL}{\rightarrowtail}
\newcommand{\ordArrR}{\twoheadrightarrow}
\newcommand{\ordCtx}{\Omega}
\newcommand{\gnab}{\text{\textexclamdown}}

% Linear logic
\newcommand{\linMult}{\otimes}
\newcommand{\linUnit}{\mathbb{1}}
\newcommand{\linArr}{\multimap}
\newcommand{\linWith}{\&}
\newcommand{\linTop}{\top}
\newcommand{\linPlus}{\oplus}
\newcommand{\linOne}{\mathbb{1}}
\newcommand{\linZero}{\mathbb{0}}
\newcommand{\linCtx}{\Delta}
\newcommand{\bang}{!}

% Intuitionistic logic
\newcommand{\iand}{\wedge}
\newcommand{\ior}{\vee}
\newcommand{\imp}{\supset}
\newcommand{\itrue}{\top}
\newcommand{\ifalse}{\bot}

% Shifts
\newcommand{\ushift}[2]{\uparrow\smash{{}^{#2}_{#1}}}
\newcommand{\dshift}[2]{\downarrow\smash{{}^{#1}_{#2}}}

% Judgments
\newcommand{\istrue}{\ \mathsf{true}}
\newcommand{\proves}{\Rightarrow}
\newcommand{\entails}{\vdash}
\newcommand{\hyp}{\ \mathsf{hyp}}
\newcommand{\conc}{\ \mathsf{conc}}
\newcommand{\verif}{\textcolor{blue}{\,\uparrow}}
\newcommand{\use}{\textcolor{red}{\,\downarrow}}

\newcommand{\DD}{\mathcal{D}}
\newcommand{\EE}{\mathcal{E}}
\newcommand{\FF}{\mathcal{F}}


\usepackage{aliascnt}

\newtheorem{theorem}{Theorem}

\newaliascnt{conjecture}{theorem}
\newtheorem{conjecture}[conjecture]{Conjecture}
\aliascntresetthe{conjecture}
\providecommand*{\conjectureautorefname}{Conjecture}
% \newtheorem{conjecture}[theorem]{Conjecture}

\newaliascnt{lemma}{theorem}
\newtheorem{lemma}[lemma]{Lemma}
\aliascntresetthe{lemma}
\providecommand*{\lemmaautorefname}{Lemma}
% \newtheorem{lemma}[theorem]{Lemma}

\newaliascnt{corollary}{theorem}
\newtheorem{corollary}[corollary]{Corollary}
\aliascntresetthe{corollary}
\providecommand*{\corollaryautorefname}{Corollary}
% \newtheorem{corollary}[theorem]{Corollary}

\newtheorem{exercise}{Exercise}
\providecommand*{\exerciseautorefname}{Exercise}

\newtheorem{discuss}{Discussion Question}
\providecommand*{\exerciseautorefname}{Discussion Question}

\newtheorem{definition}[theorem]{Definition}

% \newenvironment{proof}{\trivlist \item[\hskip \labelsep{\bf 
% Proof:}]}{\hfill$\Box$ \endtrivlist}
\newenvironment{sketch}{\trivlist \item[\hskip \labelsep{\bf 
Proof sketch:}]}{\hfill$\Box$ \endtrivlist}
\newenvironment{attempt}{\trivlist \item[\hskip \labelsep{\bf 
Proof attempt:}]}{\hfill$\Diamond$ \endtrivlist}


% Borrowed from Joseph's lec06.
\newcommand*{\bnfor}{\mid}
\newcommand*{\focsep}{\,\vert\,}
\newcommand*{\foc}[1]{\mathcolor{red}{#1}}
\newcommand*{\nat}{\operatorname{\textsf{nat}}}
\newcommand*{\s}{\operatorname{\textsf{suc}}}
\newcommand*{\z}{\operatorname{\textsf{zero}}}

\title{Lecture 10: Focused Linear Logic}
\author{Chris Martens}
\date{October 8, 2025}

\begin{document}

\maketitle

\section{Introduction}

Lecture outline:
\begin{itemize}
\item Warmup: linear logic
\item Focusing as a proof technique
\item Polarizing linear logic propositions
\item Focusing linear logic
\end{itemize}

\section{Warmup}

Prove the following sequent in linear logic:

\[
  \cdot \proves (p \with q) \lolli (r \with s) \lolli (p \tensor r)
\]

Here is one solution:

\[
  \infer[\lolli R]
{
  \cdot \proves (p \with q) \lolli (r \with s) \lolli (p \tensor r)
}{
  \infer[\lolli R]
  {p \with q \proves (r \with s) \lolli p \tensor r}
  {\infer[\with L_1]
   {p \with q, r \with s \proves p \tensor r}
   {\infer[\with L_1]
    {p, r \with s \proves p \tensor r}
    {
      \infer[\tensor R]
      {p, r \proves p \tensor r}
      {\infer[\mathsf{id}]{p \proves p}{}
      & \infer[\mathsf{id}]{r \proves r}{}
      }
    }
   }
  }
}
\]

\begin{exercise}
 Both $\tensor R$ and $\with L$ are non-invertible rules.
  Give another proof of the same proposition
  ($\cdot \proves (p \with q) \lolli (r \with s) \lolli (p \tensor r)$)
  where they are used in the other order.
\end{exercise}

\section{Focusing as a proof technique}

Last time, we claimed that linear logic was {\em more general} 
than affine logic, which permits weakening (but not contraction).
To make this claim precise, we could give an encoding of 
sequents in affine logic into linear.
Perhaps as follows:

\[
  \ulcorner \Gamma \proves^{\mathsf{aff}} A \urcorner
  =
  \Gamma \proves A \tensor \top
\]

We then need to show that this translation is sound
in the sense that translated sequents admit weakening.
This would mean: whenever $\Gamma \proves A \tensor \top$,
it is also the case that for a arbitrary proposition
$B$, we have $\Gamma, B \proves A \tensor \top$.

Unfortunately, we cannot only consider the case where
our input derivation has the form

\[
  \infer[\tensor R]
  {\Gamma \proves A \tensor \top}
  {\deduce{\Gamma_1 \proves A}{\DD_1}
  &
  \deduce{\Gamma_2 \proves \top}{\DD_2}
  &
  \Gamma = \Gamma_1, \Gamma_2}
\]

If we could, then the proof would be done by constructing
a proof of the weakened sequent as:

\[
  \infer[\tensor R]
  {\Gamma, B \proves A \tensor \top}
  {\deduce{\Gamma_1 \proves A}{\DD_1}
  &
  \infer[\top R]{\Gamma_2, B \proves \top}{}
  &
  \Gamma = \Gamma_1, \Gamma_2, B}
\]

But without further information, we would also have
to consider every possible left rule...
a focused logic drastically reduces the search
space of possible proofs, allowing us
to make this argument by considering the 
structure of the derivation at the level
of alternating focus-inversion phases.

(TODO: make this argument more carefully. The proof still isn't trivial.)

% TODO continue.

\section{Polarizing Linear Logic Propositions}

We observe that each connective has exactly one of its
{\em left rules} or {\em right rules} invertible;
this gives us an unambiguous polarization for each
connective.

Furthermore, if we choose to maintain focus as long as
possible, we can deterministically assign polarity to 
each subformula.

This gives us a way to stratify propositions into positives
and negatives, with {\em shift} operators that embed
each polarity into the other:

\begin{itemize}
  \item Positives $A^+ ::= A^+ \tensor B^+ \mid \one 
          \mid A^+ \oplus B^+ \mid \zero \mid p^+ \mid \dshift A^-$
  \item Negatives $A^- ::= A^- \with B^- \mid \top
          \mid A^+ \lolli B^- \mid p^- \mid \ushift A^+$
\end{itemize}

\section{Focused Linear Logic}

The following development is based on Simmons' ``structural focalization''
for intuitionistic logic~\cite{simmons2014structural}, including
the idea of suspended atomic propositions, which are more fully
explained in that text.

Previously, when writing down rules for a focused calculus, we 
had premises like $\Gamma \proves A \ \mathsf{stable}$
to indicate when it was permissible to move from an inversion
phase to a focus phase.
We would like to write down a calculus with a more syntactic
criterion for stability. We do this by introducing an ordered
{\em inversion context} consisting of propositions that must
be decomposed before transitioning into a focus phase.
Stability can then be defined as the inversion context being
empty and the goal proposition being ``stable'', which is a
syntactic property applying only to positive propositions
and suspended negative atoms (i.e. propositions with no further
right-inversions).

% TODO: explain suspended atomics

Inversion contexts have the following grammar and are treated as
{\em ordered sequences} rather than multisets:
\[
  \Omega ::= \cdot \mid A^+, \Omega
\]

The judgments are as follows:
\begin{itemize}
  \item $\Delta; [A^-] \proves U$ (Left focus; $U$ stable)
  \item $\Delta \proves [A^+]$ (Right focus)
  \item $\Delta; \Omega \proves A$ (Inversion)
\end{itemize}

Right inversion:
all right inversion rules require an empty inversion
context, forcing left-inversions to happen first.
\[
  \infer[\with R]
  {\Delta; \cdot \proves A^- \with B^-}
  {\Delta; \cdot \proves A^-
   &
   \Delta; \cdot \proves B^-}
\qquad
  \infer[\top R]
  {\Delta; \cdot \proves \top}{}
\]

\[
  \infer[\lolli R]
  {\Delta; \cdot \proves A^+ \lolli B^-}
  {\Delta; A^+ \proves B^-}
\]

Upshift-right and 
suspending negative atoms:
\[
 \infer[\ushift R]
 {\Delta; \cdot \proves \ushift{A^+}}
 {\Delta; \cdot \proves A^+}
\qquad
\infer[\mathsf{sus}^-]
  {\Delta; \cdot \proves p^-}
  {\Delta; \cdot \proves \langle p^- \rangle}
\]

Left inversion: decomposing positive
connectives on the left, including shifts
(which move negatives to the regular context)
and suspending positive atoms.

% tensor, 1, oplus, 0, downshift
\[
  \infer[\tensor L]
  {\Delta; A^+\tensor B^+, \Omega \proves C}
  {\Delta; A^+, B^+, \Omega \proves C}
  \qquad
  \infer[\one L]
  {\Delta; \one, \Omega \proves C}
  {\Delta; \Omega \proves C}
\]
\[
  \infer[\oplus L]
  {\Delta; A^+\oplus B^+, \Omega \proves C}
  {\Delta; A^+, \Omega \proves C
  &
  \Delta; B^+, \Omega \proves C}
  \qquad
  \infer[\zero L]
  {\Delta; \zero, \Omega \proves C}{}
\]

Downshift-left and suspended atomics:
\[
  \infer[\dshift L]
  {\Delta; \dshift A^-, \Omega \proves C}
  {\Delta, A^-; \Omega \proves C}
  \qquad
  \infer[\mathsf{sus}^+]
  {\Delta; p^+, \Omega \proves C}
  {\Delta, \langle p^+ \rangle; \Omega \proves C}
\]

Moving from inversion to focus:
\[
  \infer[\mathsf{foc} R]
  {\Delta; \cdot \proves A^+}
  {\Delta \proves [A^+]}
  \qquad
  \infer[\mathsf{foc} L]
  {\Delta, A^-; \cdot \proves U}
  {U\ \mathsf{stable}
    &
    \Delta; [A^-] \proves U}
  \qquad
  \infer{A^+\ \mathsf{stable}}{}
  \qquad
  \infer{\langle p^- \rangle\ \mathsf{stable}}{}
\]

Left focus:
% lolli left, with left, upshift, neg atom
\[
  \infer[\lolli L]
  {\Delta; [A^+ \lolli B^-] \proves U}
  {\Delta \proves [A^+]
  &
  \Delta; [B^-] \proves U}
  \qquad
  \infer[\with L_1]
  {\Delta; [A^- \with B^-] \proves U}
  {\Delta; [A^-] \proves U}
  \qquad
  {\Delta; [A^- \with B^-] \proves U}
  {\Delta; [B^-] \proves U}
\]

Upshift-left puts us back into left inversion:
\[
  \infer[\ushift L]
  {\Delta; [\ushift{A^+}] \proves U}
  {\Delta; A^+ \proves U}
\]

\[
  \infer[\mathsf{id}^-]
  {\Delta; [p^-] \proves \langle p^- \rangle}
  {\Delta = \cdot}
\]

Right focus:
% tensor right, one right, oplus right x2, downshift, pos atom
\[
  \infer[\tensor R]
  {\Delta \proves [A^+ \tensor B^+]}
  {\Delta \proves [A^+]
   &
   \Delta \proves [B^+]
  }
  \qquad
  \infer[\one R]
  {\Delta \proves [\one]}
  {\Delta = \cdot}
\]

\[
  \infer[\oplus R_1]
  {\Delta \proves [A^+ \oplus B^+]}
  {\Delta \proves [A^+]}
  \qquad
  \infer[\oplus R_2]
  {\Delta \proves [A^+ \oplus B^+]}
  {\Delta \proves [B^+]}
\]

Downshift-right puts us back into right inversion:
\[
  \infer[\dshift R]
  {\Delta \proves [\dshift A^-]}
  {\Delta; \cdot \proves A^-}
\]

\begin{exercise}
Translate your proof of
  \[
    \cdot \proves (p \with q) \lolli (r \with s) \lolli (p \tensor r)
  \]
from the previous exercise, choosing an appropriate polarization
for the atomic propositions. Explain how the structure of the proof
forces such a polarization.
\end{exercise}


\bibliographystyle{plainnat}
\bibliography{main}


\end{document}
