\documentclass{article}
\usepackage{graphicx} % Required for inserting images
\usepackage{proof-dashed,amsmath,amssymb,amsthm}
\usepackage{xcolor}
\usepackage{stmaryrd}
\usepackage{microtype}
\usepackage[utf8]{inputenc}
\newcommand{\exif}[1]{\mathsf{if}\;{#1}}
\newcommand{\exthen}[1]{\ \mathsf{then}\;{#1}}
\newcommand{\exelse}[1]{\ \mathsf{else}\;{#1}}
\newcommand{\exite}[1]{\mathsf{ite}(#1)}
\newcommand{\explus}[1]{\mathsf{plus}(#1)}
\newcommand{\extrue}{\mathsf{true}}
\newcommand{\exfalse}{\mathsf{false}}
\newcommand{\exnum}[1]{\mathsf{num}(#1)}
\newcommand{\exbool}[1]{\mathsf{bool}(#1)}
\newcommand{\exlet}[1]{\mathsf{let}(#1)}
\newcommand{\exfun}[1]{\ensuremath{\lambda{#1}}}
% \newcommand{\exapp}[2]{\mathsf{app}({#1}, {#2})}
\newcommand{\exapp}[2]{{#1}{(#2)}}
\newcommand{\exunit}{()}
\newcommand{\expair}[1]{(#1)}
\newcommand{\exlpair}[1]{\langle #1 \rangle}
\newcommand{\exinl}[1]{\mathsf{in}_1\,{#1}}
\newcommand{\exinr}[1]{\mathsf{in}_2\,{#1}}
%\newcommand{\expil}[1]{\ensuremath{\pi_1}\,{#1}}
%\newcommand{\expir}[1]{\ensuremath{\pi_2}\,{#1}}
% "dot" syntax:
\newcommand{\expil}[1]{{#1}.1}
\newcommand{\expir}[1]{{#1}.2}
\newcommand{\excase}[1]{\mathsf{case}(#1)}
\newcommand{\exsplit}[1]{\mathsf{split}(#1)}
\newcommand{\exlam}[1]{\exfun{#1}}
\newcommand{\exzero}{\mathsf{zero}}
\newcommand{\exsucc}[1]{\mathsf{succ}(#1)}
\newcommand{\exrec}[1]{\mathsf{rec}(#1)}
\newcommand{\exdiverge}{\bot}
\newcommand{\exfix}[2]{\mathsf{fix}_{#1}(#2)}
\newcommand{\exifz}[1]{\mathsf{ifz}(#1)}
\newcommand{\extyab}[1]{\Lambda{#1}}
\newcommand{\extyapp}[2]{#1[#2]}
\newcommand{\exfail}{\mathsf{fail}}
\newcommand{\extry}[1]{\mathsf{trycatch}(#1)}
\newcommand{\exalloc}[1]{\mathsf{alloc}(#1)}
\newcommand{\exassign}[2]{{#1} := {#2}}
\newcommand{\exderef}[1]{?{#1}}

% types
\newcommand{\tybool}{\mathsf{Bool}}
\newcommand{\tynum}{\mathsf{Num}}
\newcommand{\tynat}{\mathsf{nat}}
\newcommand{\tyzero}{\mathbf{0}}
% \newcommand{\tyone}{\mathbb{1}}
\newcommand{\tyone}{\mathbf{1}}
% \newcommand{\tytwo}{\mathbb{2}}
\newcommand{\tytwo}{\mathbf{2}}
\newcommand{\typrod}{\times}
\newcommand{\tyfun}{\to}
\newcommand{\tysum}{+}
\newcommand{\tywith}{\;\&\;}
\newcommand{\typarfun}{\rightharpoonup}
\newcommand{\tyall}{\forall}
\newcommand{\tyref}[1]{\mathsf{ref}(#1)}
\newcommand{\tylolli}{\multimap}
\newcommand{\tyoplus}{\oplus}
\newcommand{\tytens}{\otimes}
\newcommand{\tytop}{\top}
\newcommand{\tybang}{\,!}

\newcommand{\pzero}{\mathsf{zero}}
\newcommand{\psucc}{\mathsf{succ}}

\newcommand{\evals}{\Downarrow}
\newcommand{\isval}{\ \mathsf{value}}
\newcommand{\istype}{\ \mathsf{type}}
\newcommand{\isctx}{\ \mathsf{ctx}}

\newcommand{\steps}{\mapsto}
\newcommand{\stepstar}{\mapsto^{*}}
\newcommand{\reduces}{\longrightarrow}

% Stacks
\newcommand{\seval}{\triangleright}
\newcommand{\sret}{\triangleleft}
\newcommand{\hole}{\square}
%\DeclareUnicodeCharacter{1F635}{\failstate}
%\DeclareRobustCommand\failstate{%
%  \unskip\nobreak\thinspace\textemdash\allowbreak\thinspace\ignorespaces}
\newcommand{\failstate}{\blacktriangleleft}

%% Program equivalence
\newcommand{\iso}{\simeq}
% \newcommand{\eequiv}{\cong}
\newcommand{\eequiv}{\approx}
\newcommand{\vequiv}{\sim}

% Semantics
\newcommand{\mathnat}{\mathbf{N}}
\newcommand{\encode}[1]{\ulcorner #1\urcorner}
\newcommand{\vsem}[1]{\llbracket #1 \rrbracket}
\newcommand{\rel}[1]{\mathbf{Rel}(#1)}

% Symbol reference: https://milde.users.sourceforge.net/LUCR/Math/mathpackages/amssymb-symbols.pdf

% Ordered logic
\newcommand{\ordMult}{\bullet}
\newcommand{\ordUnit}{\epsilon}
\newcommand{\ordArrL}{\rightarrowtail}
\newcommand{\ordArrR}{\twoheadrightarrow}
\newcommand{\ordCtx}{\Omega}
\newcommand{\gnab}{\text{\textexclamdown}}

% Linear logic
\newcommand{\linMult}{\otimes}
\newcommand{\linUnit}{\mathbb{1}}
\newcommand{\linArr}{\multimap}
\newcommand{\linWith}{\&}
\newcommand{\linTop}{\top}
\newcommand{\linPlus}{\oplus}
\newcommand{\linOne}{\mathbb{1}}
\newcommand{\linZero}{\mathbb{0}}
\newcommand{\linCtx}{\Delta}
\newcommand{\bang}{!}

% Intuitionistic logic
\newcommand{\iand}{\wedge}
\newcommand{\ior}{\vee}
\newcommand{\imp}{\supset}
\newcommand{\itrue}{\top}
\newcommand{\ifalse}{\bot}

% Shifts
\newcommand{\ushift}[2]{\uparrow\smash{{}^{#2}_{#1}}}
\newcommand{\dshift}[2]{\downarrow\smash{{}^{#1}_{#2}}}

% Judgments
\newcommand{\istrue}{\ \mathsf{true}}
\newcommand{\proves}{\Rightarrow}
\newcommand{\entails}{\vdash}
\newcommand{\hyp}{\ \mathsf{hyp}}
\newcommand{\conc}{\ \mathsf{conc}}
\newcommand{\verif}{\textcolor{blue}{\,\uparrow}}
\newcommand{\use}{\textcolor{red}{\,\downarrow}}

\newcommand{\DD}{\mathcal{D}}
\newcommand{\EE}{\mathcal{E}}
\newcommand{\FF}{\mathcal{F}}


\usepackage{aliascnt}

\newtheorem{theorem}{Theorem}

\newaliascnt{conjecture}{theorem}
\newtheorem{conjecture}[conjecture]{Conjecture}
\aliascntresetthe{conjecture}
\providecommand*{\conjectureautorefname}{Conjecture}
% \newtheorem{conjecture}[theorem]{Conjecture}

\newaliascnt{lemma}{theorem}
\newtheorem{lemma}[lemma]{Lemma}
\aliascntresetthe{lemma}
\providecommand*{\lemmaautorefname}{Lemma}
% \newtheorem{lemma}[theorem]{Lemma}

\newaliascnt{corollary}{theorem}
\newtheorem{corollary}[corollary]{Corollary}
\aliascntresetthe{corollary}
\providecommand*{\corollaryautorefname}{Corollary}
% \newtheorem{corollary}[theorem]{Corollary}

\newtheorem{exercise}{Exercise}
\providecommand*{\exerciseautorefname}{Exercise}

\newtheorem{discuss}{Discussion Question}
\providecommand*{\exerciseautorefname}{Discussion Question}

\newtheorem{definition}[theorem]{Definition}

% \newenvironment{proof}{\trivlist \item[\hskip \labelsep{\bf 
% Proof:}]}{\hfill$\Box$ \endtrivlist}
\newenvironment{sketch}{\trivlist \item[\hskip \labelsep{\bf 
Proof sketch:}]}{\hfill$\Box$ \endtrivlist}
\newenvironment{attempt}{\trivlist \item[\hskip \labelsep{\bf 
Proof attempt:}]}{\hfill$\Diamond$ \endtrivlist}

\newcommand{\bnfor}{\ |\ }
\newcommand{\focsep}{\,|\,}
\newcommand{\hlbox}[1]{\colorbox{yellow}{$#1$}}

\title{Lecture 6: Intro to Focusing}
\author{Joseph Rotella}
\date{September 24, 2025}

\begin{document}

\maketitle

\section{Introduction}

Lecture outline:
\begin{itemize}
    \item Invertibility
    \item Focusing as a proof search strategy
    \item Focused sequent calculus
    \item Quantifiers
\end{itemize}

\section{Defining Invertibility}

\subsection{Sequent Calculus, Revisited}

At the end of the last lecture, we were considering which rule we should first apply in a proof of
the following sequent:
\begin{equation}
  \label{eq:6-multi-choice}
  p \iand q \imp r, p \iand s \proves p \imp r
\end{equation} 
We could apply ${\imp} L$, ${\imp} R$, or one of two ${\iand} L$ rules. However, there is a certain
difference in kind between the first and latter two rules: if we apply ${\imp} L$, we might (and,
in fact, will) end up discovering that we can't prove $p \iand q$, leading us to a dead end in our
proof, even though we might be able to obtain $r$ in some other way. On the other hand, destructing
one of the conjunctions or moving $p$ to the antecedent doesn't force us to make a commitment to a
certain (possibly erroneous) direction in our proof---applying these rules ``can't hurt.''

\subsection{Invertibility}

The intuitive notion of a ``non-committal'' rule is made precise by the concept of
\emph{invertibility}. We say that a rule is \emph{invertible} if its conclusion implies all of its
premises. This captures our previous intuition of rules that do not ``lose information'' or make us
commit to a ``narrower path'' within our proof, since we can always recover the sequent from which
we started.

We illustrate invertibility with some examples. An example of an invertible rule is the conjunction
right rule:

\[
  \infer[\iand R]
  {\Gamma \proves A \iand B}
  {\Gamma \proves A
  &
  \Gamma \proves B}
\]

To show that this is so, we must show that each of the rule's premises follows from its
conclusion, i.e., that both of the following rules are admissible:

\[
  \infer-
  {\Gamma \proves A}
  {\Gamma \proves A \iand B}
  \qquad
  \infer-
  {\Gamma \proves B}
  {\Gamma \proves A \iand B}
\]

We show the first derivation only, since the second is symmetric. Using cut, we obtain $A \iand B$
in our antecedent; the appropriate conjunction left rule then allows us to obtain $A$:

\[
  \infer-[\mathsf{cut}]
  {\Gamma \proves A}
  {\deduce{\Gamma \proves A \iand B}{}
   &
   \infer[\iand L_1]
   {\Gamma, A \iand B \proves A}
   {\infer[\mathsf{id}_A]
     {\Gamma, A \iand B, A \proves A}
     {}
   }
  }
\]

As we intuited in the initial example (\ref{eq:6-multi-choice}), an example of a \emph{non}-invertible
rule is ${\imp} L$:

\[
  \infer[{\imp} L]
  {\Gamma, A \imp B \proves C}
  {\Gamma, A \imp B \proves A
  &
  \Gamma, A \imp B, B \proves C}
\]

Were this rule invertible, the following rule (from the first premise) would be admissible:
\[
  \infer-[\textcolor{red}{\text{(Invalid)}}]
  {\Gamma, A \imp B \proves A}
  {\Gamma, A \imp B \proves C}
\]
Since the provability of $C$ in the premise need not have any bearing on the provability of $A$ in
the conclusion, this rule effectively asserts that an implication's antecedent is provable from the
implication itself, which is not generally true. Consider the case where $A$ is $\bot$, $B$ is
$\top$, and $C$ is provable: we have that $\bot \imp \top \proves C$ is derivable, but $\bot \imp
\top \proves \bot$ is not.

Our full collection of rules is categorized by invertibility as follows:

\begin{itemize}
  \item Invertible: ${\iand} R$, ${\iand} L_1$, ${\iand} L_2$, ${\ior} L$, ${\imp} R$,
        $\mathsf{id}$, $\bot L$, $\top R$
  \item Noninvertible: $\ior R_1$, $\ior R_2$, ${\imp} L$
\end{itemize}

\begin{exercise}
Show that the implication right rule is invertible and that the disjunction right rules are not.
\end{exercise}

\section{What Is Focusing?}

A key feature of invertible rules we identified in the initial example (\ref{eq:6-multi-choice}) is
that they never need to be \emph{backtracked} during proof search---they can always be freely
applied. In contrast, as we saw with ${\imp} R$, noninvertible rules may force us to commit to a
dead-end path in our search from which we will need to backtrack.

This view motivates the following approach to proof search: we should eagerly apply invertible
rules, and delay the application of noninvertible rules until we have finished applying invertible
ones. We refine this approach into a two-phase strategy: we alternate between exhaustively applying
all invertible rules and exhaustively applying noninvertibles rules only to a particular
\emph{focused} subformula and the subformulas that arise from those applications. A rough outline
of the procedure follows; we will refine this later when we have introduced additional terminology:

\begin{enumerate}
  \item \emph{Inversion}: Apply all applicable invertible rules
  \item \emph{Focusing}: Pick a place $A$ to focus
  \item \emph{Focusing (cont.)}: Maintain focus on subformulas of $A$ arising from noninvertible
  rule applications until no more noninvertible rules apply
  \item Go back to inversion
\end{enumerate}

Focusing aims to reduce the nondeterminism present in sequent-calculus proof search. To be sure,
this strategy does not entirely eliminate nondeterminism and the attendant possibility of needing to
backtrack: at any point during the focusing phase, we may apply an erroneous noninvertible rule that
leads us to a dead-end branch of the search tree. Nonetheless, it confines nondeterministic (and
potentially erroneous) choices to the focusing phase, as one can think of rules in the inversion
phase as being applied ``in parallel'' and thus deterministically (in practice, we may fix an
ordering in which to apply invertible rules).

Furthermore, this search procedure is complete for the intuitionistic sequent calculus. Since we
will realize our focused proof search through an alternative calculus of sequents $\Gamma
\proves^{\mathsf{foc}} A$ that enforces focusing, we state (but do not prove) this completeness
property as follows:

\begin{proposition}[Completeness of Focusing]
If $\Gamma \proves A$, then $\Gamma \proves^{\mathsf{foc}} A$.
\end{proposition}

To motivate our development of a focused sequent calculus, consider the following informal example
of applying the above strategy to deriving the sequent
$p \iand q \imp r, p \iand s \proves q \imp r$:

\[
  \infer-[\land L_1, \land L_2, {\imp} R]
    {p \iand q \imp r, p \iand s \proves q \imp r}
    {\infer[{\imp} L]
      {\Gamma := \hlbox{p \iand q \imp r}, p \iand s, p, s, q \proves r}
      {\infer-[\text{blur}]
        {\Gamma \proves \hlbox{p \iand q}}
        {\infer[\iand R]
          {\Gamma \proves p \iand q}
          {\infer[\mathsf{id}_p]{\Gamma \proves p}{}}}
       \qquad
       \infer-[\text{blur}]
        {\Gamma, \hlbox{r} \proves r}
        {\infer[\mathsf{id}_r]{\Gamma, r \proves r}{}}}}
\]

We begin with the inversion phase, consolidated into the first inference. With no applicable
invertible rules remaining, we apply the noninvertible implication left rule, focusing $p \land q
\imp r$ and the subformulas arising therefrom (indicated with highlighting). Neither subsequent
sequent admits a noninvertible rule application, so we \emph{blur}---i.e., ``drop focus''---and once
more apply invertible rules, which in each case discharges our remaining obligations.\footnote{In
the derivation, we write ``blur'' in serif type and with a dashed inference line to indicate its
informality; our actual blurring rules will be defined as part of a focused sequent calculus.}

\section{A Focused Sequent Calculus}

\subsection{Polarity}

For our development of a focused sequent calculus, it will be helpful to think about
invertibility properties at the level of connectives rather than individual rules: when do
(non)invertible rules apply to a formula with a given connective on a given side of a sequent?
\emph{Polarity} serves to categorize connectives in this manner.

We say that a connective is \emph{positive} if its left rules are invertible, and we say that it is
\emph{negative} if its right rules are invertible. We list the polarities of each connective below
(we noted in class that one could argue that $\bot$ is vacuously negative, but, as the unit of
disjunction---which is positive---it is naturally treated as positive):

\begin{table}[h]
  \centering
  \begin{tabular}{c|c|c}
    & Positive & Negative \\
    \hline
    $\iand$ & \checkmark & \checkmark \\
    $\ior$ & \checkmark & \\
    $\imp$ & & \checkmark \\
    $\top$ & $\checkmark$ & \checkmark \\
    $\bot$ & \checkmark &
  \end{tabular}
\end{table}

While truth and conjunction are both bipolar, it will be desirable for our development of a focused
sequent calculus to assign to every connective a unique polarity. This is because we will treat
connectives differently based on their polarity, and having to choose among ambiguous polarities for
a connective introduces a source of nondeterminism---the very thing we are seeking to minimize---%
into our system. Therefore, in the following development, we will treat truth and conjunction as
negative (the alternative is considered in an exercise).

Similarly, propositional atoms are not \emph{a priori} endowed with a particular polarity. For the
moment, we will opt to treat them as negative, though a development with positively polarized atoms
is also possible; we will see later the consequences of this decision.

\begin{exercise}
Think about what it would look like to have a connective that is \emph{neither} positive nor
negative. Why would this be interesting? Try to design such a connective, and discuss your findings.
\end{exercise}

\subsection{Judgments, Grammar, and Inference Rules}

Our polarized sequent calculus comprises three forms of judgment, allowing us to specify when a
sequent is in focus:
\[
  \Gamma \proves A \text{ (inversion)} \qquad
  \Gamma \focsep [A] \proves C \text{ (left focus)} \qquad
  \Gamma \proves [C] \text{ (right focus)}
\]
We pass between these judgments using \emph{focusing} and \emph{blurring} rules, defined shortly. We
also informally define a judgment $\Gamma \proves A \textsf{ stable}$ that says that $\Gamma \proves
A$ has had all applicable invertible rules applied.

Connectives' polarities are directly encoded in our grammar. In a future lecture, we will refine
this grammar to eliminate the need for unpolarized formulas.
\begin{alignat*}{2}
  &A, B ::= A^+ \bnfor A^- &&\text{formulas}\\
  &A^+, B^+ ::= A \lor B \bnfor \bot &&\text{positive connectives}\\
  &A^-, B^- ::= A \land B \bnfor A \imp B \bnfor \top \bnfor p \quad &&\text{negative connectives}
\end{alignat*}

Equipped with the vocabulary of polarity, we can rephrase our earlier proof-search ``recipe'':

\begin{enumerate}
  \item \emph{Inversion}: Apply invertible rules \underline{until a stable sequent is reached}
  \item \emph{Focusing}: Pick a place $A$ to focus
  \item \emph{Focusing (cont.)}: Maintain focus on subformulas of $A$ arising from noninvertible
  rule applications until \underline{reaching a sequent of the form $\Gamma \proves [A^-]$ or}
  \underline{$\Gamma \focsep [A^+] \proves C$}
  \item Go back to inversion
\end{enumerate}

Notice that our rewrite of step (3) is justified by the fact that negative connectives have
invertible right rules, so that a focused formula with a negative connective on the right-hand side
will have no applicable noninvertible rules, and symmetrically for positive connectives on the left.

The rules of our focused sequent calculus, shown below, differentiate and provide mechanisms for
passing between these two phases of proof search. Notice that the $\mathsf{foc} L$ rule now ensures
that focused formulas in the antecedent are retained in the ``unfocused'' portion of the context, so
we drop principal-formula propagation from our noninvertible left rules. The new and modified rules
are highlighted:

\begin{equation*}
\begin{gathered}
  \infer[\hlbox{\mathsf{foc} R}]
    {\Gamma \proves A^+}
    {\Gamma \proves A^+ \textsf{ stable}
    &
    \Gamma \proves [A^+]}
  \qquad
  \infer[\hlbox{\mathsf{foc} L}]
    {\Gamma, A^- \proves C}
    {\Gamma, A^- \proves C \textsf{ stable}
    &
    \Gamma, A^- \focsep [A^-] \proves C}
  \\
  \infer[\iand R]
    {\Gamma \proves A \iand B}
    {\Gamma \proves A
      &
      \Gamma \proves B}
  \qquad
  \infer[\hlbox{\iand L_1}]
    {\Gamma \focsep [A \iand B] \proves C}
    {\Gamma \focsep [A] \proves C}
  \qquad
  \infer[\hlbox{\iand L_2}]
    {\Gamma \focsep [A \iand B] \proves C}
    {\Gamma \focsep [B] \proves C}
  \\
  \infer[\hlbox{\ior R_1}]
    {\Gamma \proves [A \ior B]}
    {\Gamma \proves [A]}
  \qquad
  \infer[\hlbox{\ior R_2}]
    {\Gamma \proves [A \ior B]}
    {\Gamma \proves [B]}
  \qquad
  \infer[\ior L]
    {\Gamma, A \ior B \proves C}
    {\Gamma, A \ior B, A \proves C
    &
    \Gamma, A \ior B, B \proves C
    }
  \\
  \infer[{\imp} R]
    {\Gamma \proves A \imp B}
    {\Gamma, A \proves B}
  \qquad
  \infer[\hlbox{{\imp} L}]
    {\Gamma, A \imp B \proves C}
    {\Gamma \proves [A]
    &
    \Gamma \focsep [B] \proves C}
  \\
  \infer[\top R]
    {\Gamma \proves \top}{}
  \qquad
  \infer[\bot L]
    {\Gamma, \bot \proves C} {}
  \\
  \infer[\hlbox{\mathsf{blur} L}]
    {\Gamma \focsep [A^+] \proves C}
    {\Gamma, A^+ \proves C}
  \qquad
  \infer[\hlbox{\mathsf{blur} R}]
    {\Gamma \proves [A^-]}
    {\Gamma \proves A^-}
  \\
  \infer[\hlbox{\mathsf{id}^-}]
    {\Gamma \focsep [p] \proves p}
    {}
\end{gathered}
\end{equation*}


\begin{exercise}
Write the rules for conjunction treated as a positive connective.
\end{exercise}

As an example, we provide a focused proof of $p, p \imp q, q \imp r, s \imp r \proves r$, eliding
stability judgments (and letting $\Gamma := p, p \imp q, q \imp r, s \imp r$):
\begin{equation}
  \begin{aligned}
  \label{eq:6-focused-proof}
  \infer[\mathsf{foc} L]
    {p, p \imp q, q \imp r, s \imp r \proves r}
    {\infer[{\imp} L]
      {\Gamma \focsep [q \imp r] \proves r}
      {\infer[\mathsf{blur} R]
        {\Gamma \proves [q]}
        {\infer[\mathsf{foc} L]
          {\Gamma \proves q}
          {\infer[{\imp} L]
            {\Gamma \focsep [p \imp q] \proves q}
            {\infer[\mathsf{blur} R]
              {\Gamma \proves [p]}
              {\infer[\mathsf{foc} L]
                {\Gamma \proves p}
                {\infer[\mathsf{id}^-]
                  {\Gamma \focsep [p] \proves p}
                  {}}}
             &
             \infer[\mathsf{id}^-]
              {\Gamma \focsep [q] \proves q}
              {}
            }
          }}
       &
       \infer[{\mathsf{id}^-}]{\Gamma \focsep [r] \proves r}{}
      }
    }
  \end{aligned}
\end{equation}

\begin{exercise}
  Give a focused proof of the following sequent:
  \[ (p \imp q) \ior (p \imp r) \proves p \imp (q \ior r) \]
  
  Is there more than one possible proof? If so, explain why.
\end{exercise}

Notice that the above proof (\ref{eq:6-focused-proof}), structured as such, would not succeed if our
atoms were positively polarized. In particular, we would be unable to derive $\Gamma \proves [q]$
after the first ${\imp} L$ step, since $\mathsf{blur} R$ would be inapplicable with the positively
polarized $q$ on the right-hand side. In other words, with positively polarized atoms, we must
already have an atom in the antecedent before it arises in the consequent of a sequent. This is
reflected in the identity rule for positively polarized atoms:

\[
\infer[\mathsf{id}^+]
  {\Gamma, p \proves [p]}
  {}
\]

To restructure the proof (\ref{eq:6-focused-proof}) for positively polarized atoms, we would need to
ensure that, whenever we apply one of the implication left rules, the antecedent of the implication
is already present in our context. Accordingly, rather than beginning by reasoning backwards from
the consequent $r$ in our derivation, we would instead need to reason forward through the sequence
of implications starting with $p \imp q$, whose antecedent $p$ is already known, then proceeding to
$q \imp r$ once $q$ has been added to the left-hand side. In other words, positively polarizing
atoms yields a \emph{forward chaining} proof structure, proceeding from hypotheses to goal; the
negatively polarized approach above yields \emph{backward chaining}, working from goal to
hypotheses.

\section{Quantifiers}

We now move from propositional to first-order logic. In this lecture, we will begin by adding the
universal quantifier to our logic; next time, we will consider existentials and focused inference
rules for quantifiers.

To support quantification, we must extend our grammar with objects $a, b, c$, types $\tau$, and a
judgment $t : \tau$ that ascribes to $t$ the type $\tau$. Quantified variables are typed: we write
$\forall x : \tau.\, A_x$ for universal quantification, where the subscript $x$ denotes that that
variable may appear free in $A$. The inference rules for universal quantification are as follows:

\[
\infer[\forall R]
  {\Gamma \proves \forall x : \tau.\, A}
  {\Gamma, a : \tau \proves A[a/x]}
\qquad
\infer[\forall L]
  {\Gamma, \forall x : \tau.\, A \proves C}
  {\Gamma, \forall x : \tau.\, A \proves t : \tau
   &
   \Gamma, \forall x : \tau.\, A, A[t/x] \proves C}
\]

Intuitively, to show $\forall x : \tau.\, A$ on the right, we must show $A$ for an arbitrary object
of the appropriate type; to use $\forall x : \tau.\, A$ on the left, we can exploit the fact that 
$A$ holds for any particular object of type $\tau$.

If one squints, the broad shapes of these rules look reminiscent of those for implication. And,
indeed, one can show that, like implication, the universal quantifier is negative.

\begin{exercise}
Show that the universal quantifier is negative.
\end{exercise}

\bibliographystyle{plainnat}
\bibliography{main}

\end{document}
