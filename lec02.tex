\documentclass{article}
\usepackage{graphicx} % Required for inserting images
\usepackage{proof-dashed,amsmath,amssymb,amsthm}
\usepackage{xcolor}
\usepackage{stmaryrd}
\usepackage{microtype}
\usepackage[utf8]{inputenc}
\newcommand{\exif}[1]{\mathsf{if}\;{#1}}
\newcommand{\exthen}[1]{\ \mathsf{then}\;{#1}}
\newcommand{\exelse}[1]{\ \mathsf{else}\;{#1}}
\newcommand{\exite}[1]{\mathsf{ite}(#1)}
\newcommand{\explus}[1]{\mathsf{plus}(#1)}
\newcommand{\extrue}{\mathsf{true}}
\newcommand{\exfalse}{\mathsf{false}}
\newcommand{\exnum}[1]{\mathsf{num}(#1)}
\newcommand{\exbool}[1]{\mathsf{bool}(#1)}
\newcommand{\exlet}[1]{\mathsf{let}(#1)}
\newcommand{\exfun}[1]{\ensuremath{\lambda{#1}}}
% \newcommand{\exapp}[2]{\mathsf{app}({#1}, {#2})}
\newcommand{\exapp}[2]{{#1}{(#2)}}
\newcommand{\exunit}{()}
\newcommand{\expair}[1]{(#1)}
\newcommand{\exlpair}[1]{\langle #1 \rangle}
\newcommand{\exinl}[1]{\mathsf{in}_1\,{#1}}
\newcommand{\exinr}[1]{\mathsf{in}_2\,{#1}}
%\newcommand{\expil}[1]{\ensuremath{\pi_1}\,{#1}}
%\newcommand{\expir}[1]{\ensuremath{\pi_2}\,{#1}}
% "dot" syntax:
\newcommand{\expil}[1]{{#1}.1}
\newcommand{\expir}[1]{{#1}.2}
\newcommand{\excase}[1]{\mathsf{case}(#1)}
\newcommand{\exsplit}[1]{\mathsf{split}(#1)}
\newcommand{\exlam}[1]{\exfun{#1}}
\newcommand{\exzero}{\mathsf{zero}}
\newcommand{\exsucc}[1]{\mathsf{succ}(#1)}
\newcommand{\exrec}[1]{\mathsf{rec}(#1)}
\newcommand{\exdiverge}{\bot}
\newcommand{\exfix}[2]{\mathsf{fix}_{#1}(#2)}
\newcommand{\exifz}[1]{\mathsf{ifz}(#1)}
\newcommand{\extyab}[1]{\Lambda{#1}}
\newcommand{\extyapp}[2]{#1[#2]}
\newcommand{\exfail}{\mathsf{fail}}
\newcommand{\extry}[1]{\mathsf{trycatch}(#1)}
\newcommand{\exalloc}[1]{\mathsf{alloc}(#1)}
\newcommand{\exassign}[2]{{#1} := {#2}}
\newcommand{\exderef}[1]{?{#1}}

% types
\newcommand{\tybool}{\mathsf{Bool}}
\newcommand{\tynum}{\mathsf{Num}}
\newcommand{\tynat}{\mathsf{nat}}
\newcommand{\tyzero}{\mathbf{0}}
% \newcommand{\tyone}{\mathbb{1}}
\newcommand{\tyone}{\mathbf{1}}
% \newcommand{\tytwo}{\mathbb{2}}
\newcommand{\tytwo}{\mathbf{2}}
\newcommand{\typrod}{\times}
\newcommand{\tyfun}{\to}
\newcommand{\tysum}{+}
\newcommand{\tywith}{\;\&\;}
\newcommand{\typarfun}{\rightharpoonup}
\newcommand{\tyall}{\forall}
\newcommand{\tyref}[1]{\mathsf{ref}(#1)}
\newcommand{\tylolli}{\multimap}
\newcommand{\tyoplus}{\oplus}
\newcommand{\tytens}{\otimes}
\newcommand{\tytop}{\top}
\newcommand{\tybang}{\,!}

\newcommand{\pzero}{\mathsf{zero}}
\newcommand{\psucc}{\mathsf{succ}}

\newcommand{\evals}{\Downarrow}
\newcommand{\isval}{\ \mathsf{value}}
\newcommand{\istype}{\ \mathsf{type}}
\newcommand{\isctx}{\ \mathsf{ctx}}

\newcommand{\steps}{\mapsto}
\newcommand{\stepstar}{\mapsto^{*}}
\newcommand{\reduces}{\longrightarrow}

% Stacks
\newcommand{\seval}{\triangleright}
\newcommand{\sret}{\triangleleft}
\newcommand{\hole}{\square}
%\DeclareUnicodeCharacter{1F635}{\failstate}
%\DeclareRobustCommand\failstate{%
%  \unskip\nobreak\thinspace\textemdash\allowbreak\thinspace\ignorespaces}
\newcommand{\failstate}{\blacktriangleleft}

%% Program equivalence
\newcommand{\iso}{\simeq}
% \newcommand{\eequiv}{\cong}
\newcommand{\eequiv}{\approx}
\newcommand{\vequiv}{\sim}

% Semantics
\newcommand{\mathnat}{\mathbf{N}}
\newcommand{\encode}[1]{\ulcorner #1\urcorner}
\newcommand{\vsem}[1]{\llbracket #1 \rrbracket}
\newcommand{\rel}[1]{\mathbf{Rel}(#1)}

% Symbol reference: https://milde.users.sourceforge.net/LUCR/Math/mathpackages/amssymb-symbols.pdf

% Ordered logic
\newcommand{\ordMult}{\bullet}
\newcommand{\ordUnit}{\epsilon}
\newcommand{\ordArrL}{\rightarrowtail}
\newcommand{\ordArrR}{\twoheadrightarrow}
\newcommand{\ordCtx}{\Omega}
\newcommand{\gnab}{\text{\textexclamdown}}

% Linear logic
\newcommand{\linMult}{\otimes}
\newcommand{\linUnit}{\mathbb{1}}
\newcommand{\linArr}{\multimap}
\newcommand{\linWith}{\&}
\newcommand{\linTop}{\top}
\newcommand{\linPlus}{\oplus}
\newcommand{\linOne}{\mathbb{1}}
\newcommand{\linZero}{\mathbb{0}}
\newcommand{\linCtx}{\Delta}
\newcommand{\bang}{!}

% Intuitionistic logic
\newcommand{\iand}{\wedge}
\newcommand{\ior}{\vee}
\newcommand{\imp}{\supset}
\newcommand{\itrue}{\top}
\newcommand{\ifalse}{\bot}

% Shifts
\newcommand{\ushift}[2]{\uparrow\smash{{}^{#2}_{#1}}}
\newcommand{\dshift}[2]{\downarrow\smash{{}^{#1}_{#2}}}

% Judgments
\newcommand{\istrue}{\ \mathsf{true}}
\newcommand{\proves}{\Rightarrow}
\newcommand{\entails}{\vdash}
\newcommand{\hyp}{\ \mathsf{hyp}}
\newcommand{\conc}{\ \mathsf{conc}}
\newcommand{\verif}{\textcolor{blue}{\,\uparrow}}
\newcommand{\use}{\textcolor{red}{\,\downarrow}}

\newcommand{\DD}{\mathcal{D}}
\newcommand{\EE}{\mathcal{E}}
\newcommand{\FF}{\mathcal{F}}


\usepackage{aliascnt}

\newtheorem{theorem}{Theorem}

\newaliascnt{conjecture}{theorem}
\newtheorem{conjecture}[conjecture]{Conjecture}
\aliascntresetthe{conjecture}
\providecommand*{\conjectureautorefname}{Conjecture}
% \newtheorem{conjecture}[theorem]{Conjecture}

\newaliascnt{lemma}{theorem}
\newtheorem{lemma}[lemma]{Lemma}
\aliascntresetthe{lemma}
\providecommand*{\lemmaautorefname}{Lemma}
% \newtheorem{lemma}[theorem]{Lemma}

\newaliascnt{corollary}{theorem}
\newtheorem{corollary}[corollary]{Corollary}
\aliascntresetthe{corollary}
\providecommand*{\corollaryautorefname}{Corollary}
% \newtheorem{corollary}[theorem]{Corollary}

\newtheorem{exercise}{Exercise}
\providecommand*{\exerciseautorefname}{Exercise}

\newtheorem{discuss}{Discussion Question}
\providecommand*{\exerciseautorefname}{Discussion Question}

\newtheorem{definition}[theorem]{Definition}

% \newenvironment{proof}{\trivlist \item[\hskip \labelsep{\bf 
% Proof:}]}{\hfill$\Box$ \endtrivlist}
\newenvironment{sketch}{\trivlist \item[\hskip \labelsep{\bf 
Proof sketch:}]}{\hfill$\Box$ \endtrivlist}
\newenvironment{attempt}{\trivlist \item[\hskip \labelsep{\bf 
Proof attempt:}]}{\hfill$\Diamond$ \endtrivlist}

\newcommand{\impliesR}{\implies_{R}}
\newcommand{\impliesE}{\implies_{E}}
\newcommand{\caseof}[5]{\mathbf{case}~(#1)~\mathbf{of}~(#2 \Rightarrow #3 \mid #4 \Rightarrow #5)}

\title{Lecture 2: Harmony}
\author{Luis Garcia}
\date{\today}

\begin{document}

\maketitle

\section{Introduction}
Lecture outline:
\begin{itemize}
    \item Harmony
    \item Normalization for the STLC
    \item Global Soundness and Completeness
    \item Sequent Calculus
\end{itemize}

We want to show that our logic ``makes sense". In other words, we want soundness. Last lecture, we posited that a definition of soundness can be showing that it is impossible to prove false. However, that's a bit of a narrow view. What we want to show is that our elimination rules aren't ``too strong"---that is, that they don't produce new information we don't already have. Dual to this is showing that our elimination rules aren't ``too weak". We will explore what this means with the concept of \textit{harmony}, realized by the notions of \textit{local soundness} and \textit{local completeness}.

\section{Local Soundness and Completeness}
Local soundness and completeness are heuristics for logic design\footnote{Side note: it does not work for \textit{every} logic. However, it's still a good exercise to do when designing a logic.}. Local soundness can be witnessed by a \textit{local reduction}, which is achieved by constructing evidence for a conclusion from evidence for its premises. Here's an example with conjunction:

\[
    \begin{array}{ccc}
         \infer[\iand E_1]
        {\Gamma \vdash A \istrue}
        {
            \infer[\iand I]
            {\Gamma \vdash A \iand B \istrue}
            {
                \deduce{\Gamma \vdash A \istrue}{\mathcal{D}}
                &
                \deduce{\Gamma \vdash B \istrue}{\mathcal{E}}
            }
        }
        & 
        \impliesR
        &
        \deduce{\Gamma \vdash A \istrue}{\mathcal{D}}
         
    \end{array}
\]

Local completeness is witnessed by a \textit{local expansion}, which is achieved by applying the elimination rules to a judgment to recover the original judgment. Again, with conjunction:

\[
\begin{array}{ccc}
     \deduce{\Gamma \vdash A \iand B \istrue}{\mathcal{D}} &
     \impliesE &
     \infer[\iand I]
     {
        \Gamma \vdash A \iand B \istrue
     }
     {
        \infer[\iand E_1]{
            \Gamma \vdash A \istrue
        }{
            \deduce{\Gamma \vdash A \iand B \istrue}{\mathcal{D}}
        }
        \infer[\iand E_2]{
            \Gamma \vdash B \istrue
        }{
            \deduce{\Gamma \vdash A \iand B \istrue}{\mathcal{D}}
        }
     }
\end{array}
\]

We can complete the same exercise for other connectives to show that they are harmonious. Disjunction is a bit more complicated than conjunction. It also gives us a chance to see what happens when local soundness or completeness fails. Consider a ``bad" rule for elimination rule for disjunction:

\[
    \infer[\ior E?]{
        \Gamma \vdash A \istrue
    }{
        \Gamma \vdash A \ior B \istrue
    }
\]

Checking for local soundness, we can see that we cannot locally reduce an elimination followed by an introduction:

\[
    \infer[\ior E?]{
        \Gamma \vdash A \istrue
    }{
        \infer[\ior I_2]{
            \Gamma \vdash A \ior B \istrue
        }{
            \deduce{\Gamma \vdash B \istrue}{\mathcal{D}}
        }
    }
\]

Our elimination rule is too strong: it allows us to conclude $A \istrue$ even though we have no evidence to prove it. To see the contrast with the correct rule, we need to employ the \textit{substitution principle}.

\section{Substitution}

\begin{definition}[Substitution Principle]
    If $\Gamma, A \istrue \vdash C \istrue$ and $\Gamma \vdash A \istrue$, then $\Gamma \vdash C \istrue$.
\end{definition}

This is a fundamental principle of natural deduction that corresponds to variable substitutions in a lambda calculus. Let's see how. The local reduction of disjunction is written below. There are actually two local reductions we could do. The case where we use $\ior I_2$ is symmetric to the one we show here. We write $subst(\mathcal{E}, \mathcal{D})$ to show that we are applying the substitution principle on the derivations.

\[
\begin{array}{ccc}
     \infer[\ior E]{
        \Gamma \vdash C \istrue
     } {
        \infer[\ior I_1]{
            \Gamma \vdash A \ior B \istrue
        }{
            \deduce{\Gamma \vdash A \istrue}{\mathcal{D}}
        }
        &
        \deduce{\Gamma, A\istrue \vdash C\istrue}{\mathcal{E}}
        &
        \deduce{\Gamma, B\istrue \vdash C\istrue}{\mathcal{F}}
     }
     & 
     \impliesR
     &  
     \deduce{\Gamma \vdash C\istrue}{subst(\mathcal{E}, \mathcal{D})}
\end{array}
\]

For completeness, here's local expansion:

\[
    \begin{array}{ccc}
        \deduce{\Gamma \vdash A \ior B \istrue}{\mathcal{D}}
        &  
        \impliesE
        &
        \infer[\ior E]{
            \Gamma \vdash A \ior B \istrue
        }{
            \deduce{\Gamma \vdash A \ior B \istrue}{\mathcal{D}}
            &
            \infer[\ior I_1]{
                \Gamma, A \vdash A \ior B \istrue
            }{
                \infer[hyp]{
                    \Gamma, A\istrue \vdash A\istrue
                }{}
            }
            &
            \infer[\ior I_2]{
                \Gamma, B \vdash A \ior B \istrue
            }{
                \infer[hyp]{
                    \Gamma, B\istrue \vdash B\istrue
                }{}
            }
        }
    \end{array}
\]

Now, how does the substitution principle here relate to substitution in programming? Recall that we can condense a proof tree into a proof term in the lambda calculus. For the above tree for the local soundness demonstration, we can write the term:
\[
    \caseof{inl~M}{x}{N}{y}{P}
\]
where $M : A\ior B$, $x : A$, and $y : B$. Then, the local reduction tells us that we have the following
\[
    [M / x]N.
\]
From the world of programming languages, this is a simple $\beta$-reduction rule!
\[
    \caseof{inl~M}{x}{N}{y}{P} \mapsto_\beta [M / x]N
\]

Similarly, local expansion corresponds to $\eta$-expansion rules:
\[
    M \mapsto_\eta \langle inl~M, inr~M\rangle
\]
provided that $M : A \ior B$.

\begin{exercise}
    Show that the rules for implication are locally sound and complete, and show how they relate to $\beta$-reduction and $\eta$-expansion.
\end{exercise}

\section{Normalization}
By thinking about soundness, we are also working toward a sense for what \textit{normalization} means. The vibe is ``there are no detours or circularities in our proofs". By detours, we mean exactly the kinds of things we are doing to show local soundness and completeness: pointless invocations of rules that yield no new information from our hypotheses. 


\end{document}
