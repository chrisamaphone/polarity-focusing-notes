\documentclass{article}
\usepackage{graphicx} % Required for inserting images
\usepackage{proof-dashed,amsmath,amssymb,amsthm}
\usepackage{xcolor}
\usepackage{stmaryrd}
\usepackage{microtype}
\usepackage[utf8]{inputenc}
\newcommand{\exif}[1]{\mathsf{if}\;{#1}}
\newcommand{\exthen}[1]{\ \mathsf{then}\;{#1}}
\newcommand{\exelse}[1]{\ \mathsf{else}\;{#1}}
\newcommand{\exite}[1]{\mathsf{ite}(#1)}
\newcommand{\explus}[1]{\mathsf{plus}(#1)}
\newcommand{\extrue}{\mathsf{true}}
\newcommand{\exfalse}{\mathsf{false}}
\newcommand{\exnum}[1]{\mathsf{num}(#1)}
\newcommand{\exbool}[1]{\mathsf{bool}(#1)}
\newcommand{\exlet}[1]{\mathsf{let}(#1)}
\newcommand{\exfun}[1]{\ensuremath{\lambda{#1}}}
% \newcommand{\exapp}[2]{\mathsf{app}({#1}, {#2})}
\newcommand{\exapp}[2]{{#1}{(#2)}}
\newcommand{\exunit}{()}
\newcommand{\expair}[1]{(#1)}
\newcommand{\exlpair}[1]{\langle #1 \rangle}
\newcommand{\exinl}[1]{\mathsf{in}_1\,{#1}}
\newcommand{\exinr}[1]{\mathsf{in}_2\,{#1}}
%\newcommand{\expil}[1]{\ensuremath{\pi_1}\,{#1}}
%\newcommand{\expir}[1]{\ensuremath{\pi_2}\,{#1}}
% "dot" syntax:
\newcommand{\expil}[1]{{#1}.1}
\newcommand{\expir}[1]{{#1}.2}
\newcommand{\excase}[1]{\mathsf{case}(#1)}
\newcommand{\exsplit}[1]{\mathsf{split}(#1)}
\newcommand{\exlam}[1]{\exfun{#1}}
\newcommand{\exzero}{\mathsf{zero}}
\newcommand{\exsucc}[1]{\mathsf{succ}(#1)}
\newcommand{\exrec}[1]{\mathsf{rec}(#1)}
\newcommand{\exdiverge}{\bot}
\newcommand{\exfix}[2]{\mathsf{fix}_{#1}(#2)}
\newcommand{\exifz}[1]{\mathsf{ifz}(#1)}
\newcommand{\extyab}[1]{\Lambda{#1}}
\newcommand{\extyapp}[2]{#1[#2]}
\newcommand{\exfail}{\mathsf{fail}}
\newcommand{\extry}[1]{\mathsf{trycatch}(#1)}
\newcommand{\exalloc}[1]{\mathsf{alloc}(#1)}
\newcommand{\exassign}[2]{{#1} := {#2}}
\newcommand{\exderef}[1]{?{#1}}

% types
\newcommand{\tybool}{\mathsf{Bool}}
\newcommand{\tynum}{\mathsf{Num}}
\newcommand{\tynat}{\mathsf{nat}}
\newcommand{\tyzero}{\mathbf{0}}
% \newcommand{\tyone}{\mathbb{1}}
\newcommand{\tyone}{\mathbf{1}}
% \newcommand{\tytwo}{\mathbb{2}}
\newcommand{\tytwo}{\mathbf{2}}
\newcommand{\typrod}{\times}
\newcommand{\tyfun}{\to}
\newcommand{\tysum}{+}
\newcommand{\tywith}{\;\&\;}
\newcommand{\typarfun}{\rightharpoonup}
\newcommand{\tyall}{\forall}
\newcommand{\tyref}[1]{\mathsf{ref}(#1)}
\newcommand{\tylolli}{\multimap}
\newcommand{\tyoplus}{\oplus}
\newcommand{\tytens}{\otimes}
\newcommand{\tytop}{\top}
\newcommand{\tybang}{\,!}

\newcommand{\pzero}{\mathsf{zero}}
\newcommand{\psucc}{\mathsf{succ}}

\newcommand{\evals}{\Downarrow}
\newcommand{\isval}{\ \mathsf{value}}
\newcommand{\istype}{\ \mathsf{type}}
\newcommand{\isctx}{\ \mathsf{ctx}}

\newcommand{\steps}{\mapsto}
\newcommand{\stepstar}{\mapsto^{*}}
\newcommand{\reduces}{\longrightarrow}

% Stacks
\newcommand{\seval}{\triangleright}
\newcommand{\sret}{\triangleleft}
\newcommand{\hole}{\square}
%\DeclareUnicodeCharacter{1F635}{\failstate}
%\DeclareRobustCommand\failstate{%
%  \unskip\nobreak\thinspace\textemdash\allowbreak\thinspace\ignorespaces}
\newcommand{\failstate}{\blacktriangleleft}

%% Program equivalence
\newcommand{\iso}{\simeq}
% \newcommand{\eequiv}{\cong}
\newcommand{\eequiv}{\approx}
\newcommand{\vequiv}{\sim}

% Semantics
\newcommand{\mathnat}{\mathbf{N}}
\newcommand{\encode}[1]{\ulcorner #1\urcorner}
\newcommand{\vsem}[1]{\llbracket #1 \rrbracket}
\newcommand{\rel}[1]{\mathbf{Rel}(#1)}

% Symbol reference: https://milde.users.sourceforge.net/LUCR/Math/mathpackages/amssymb-symbols.pdf

% Ordered logic
\newcommand{\ordMult}{\bullet}
\newcommand{\ordUnit}{\epsilon}
\newcommand{\ordArrL}{\rightarrowtail}
\newcommand{\ordArrR}{\twoheadrightarrow}
\newcommand{\ordCtx}{\Omega}
\newcommand{\gnab}{\text{\textexclamdown}}

% Linear logic
\newcommand{\linMult}{\otimes}
\newcommand{\linUnit}{\mathbb{1}}
\newcommand{\linArr}{\multimap}
\newcommand{\linWith}{\&}
\newcommand{\linTop}{\top}
\newcommand{\linPlus}{\oplus}
\newcommand{\linOne}{\mathbb{1}}
\newcommand{\linZero}{\mathbb{0}}
\newcommand{\linCtx}{\Delta}
\newcommand{\bang}{!}

% Intuitionistic logic
\newcommand{\iand}{\wedge}
\newcommand{\ior}{\vee}
\newcommand{\imp}{\supset}
\newcommand{\itrue}{\top}
\newcommand{\ifalse}{\bot}

% Shifts
\newcommand{\ushift}[2]{\uparrow\smash{{}^{#2}_{#1}}}
\newcommand{\dshift}[2]{\downarrow\smash{{}^{#1}_{#2}}}

% Judgments
\newcommand{\istrue}{\ \mathsf{true}}
\newcommand{\proves}{\Rightarrow}
\newcommand{\entails}{\vdash}
\newcommand{\hyp}{\ \mathsf{hyp}}
\newcommand{\conc}{\ \mathsf{conc}}
\newcommand{\verif}{\textcolor{blue}{\,\uparrow}}
\newcommand{\use}{\textcolor{red}{\,\downarrow}}

\newcommand{\DD}{\mathcal{D}}
\newcommand{\EE}{\mathcal{E}}
\newcommand{\FF}{\mathcal{F}}


\usepackage{aliascnt}

\newtheorem{theorem}{Theorem}

\newaliascnt{conjecture}{theorem}
\newtheorem{conjecture}[conjecture]{Conjecture}
\aliascntresetthe{conjecture}
\providecommand*{\conjectureautorefname}{Conjecture}
% \newtheorem{conjecture}[theorem]{Conjecture}

\newaliascnt{lemma}{theorem}
\newtheorem{lemma}[lemma]{Lemma}
\aliascntresetthe{lemma}
\providecommand*{\lemmaautorefname}{Lemma}
% \newtheorem{lemma}[theorem]{Lemma}

\newaliascnt{corollary}{theorem}
\newtheorem{corollary}[corollary]{Corollary}
\aliascntresetthe{corollary}
\providecommand*{\corollaryautorefname}{Corollary}
% \newtheorem{corollary}[theorem]{Corollary}

\newtheorem{exercise}{Exercise}
\providecommand*{\exerciseautorefname}{Exercise}

\newtheorem{discuss}{Discussion Question}
\providecommand*{\exerciseautorefname}{Discussion Question}

\newtheorem{definition}[theorem]{Definition}

% \newenvironment{proof}{\trivlist \item[\hskip \labelsep{\bf 
% Proof:}]}{\hfill$\Box$ \endtrivlist}
\newenvironment{sketch}{\trivlist \item[\hskip \labelsep{\bf 
Proof sketch:}]}{\hfill$\Box$ \endtrivlist}
\newenvironment{attempt}{\trivlist \item[\hskip \labelsep{\bf 
Proof attempt:}]}{\hfill$\Diamond$ \endtrivlist}

\title{Lecture 1: Natural Deduction}
\author{Chris Martens}
\date{\today}

\begin{document}

\maketitle

\section{Introduction}

Lecture outline:
\begin{itemize}
    \item Propositions, connectives, judgments
    \item Natural Deduction
    \item Proof terms
    \item Counting proofs
    \item Harmony
\end{itemize}

\section{Propositions, connectives, judgments}

Judgment: $\Gamma \vdash A \istrue$

\section{Inference rules for propositional natural deduction}

(Approximately Gentzen's NJ)

Conjunction:
\[
  \infer[\iand I]
  {\Gamma \vdash A \iand B \istrue}
  {\Gamma \vdash A \istrue
  &
  \Gamma \vdash B \istrue}
\qquad
  \infer[\iand E_1]
  {\Gamma \vdash A \istrue}
  {\Gamma \vdash A \iand B \istrue}
\qquad
  \infer[\iand E_2]
  {\Gamma \vdash B \istrue}
  {\Gamma \vdash A \iand B \istrue}
\]

Disjunction:
\[
  \infer[\ior I_1]
  {\Gamma \vdash A \ior B \istrue}
  {\Gamma \vdash A \istrue}
\qquad
  \infer[\ior I_2]
  {\Gamma \vdash A \ior B \istrue}
  {\Gamma \vdash B \istrue}
\]
\[
  \infer[\ior E]
  {\Gamma \vdash C \istrue}
  {\Gamma \vdash A \ior B \istrue
   &
   \Gamma, A\istrue \vdash C \istrue
   &
   \Gamma, B\istrue \vdash C \istrue
  }
\]

(The hypothetical judgment and hypothesis rule)

\[
  \infer[\mathsf{hyp}]
  {\Gamma \vdash A\istrue}
  {A\istrue \in \Gamma}
\]

Implication:

\[
  \infer[\imp I]
  {\Gamma \vdash A \imp B \istrue}
  {\Gamma, A\istrue \vdash B \istrue}
\qquad
  \infer[\imp E]
  {\Gamma \vdash B \istrue}
  {\Gamma \vdash A \imp B \istrue
  &
  \Gamma \vdash A \istrue}
\]

Truth and Falsehood:
\[
  \infer[\top I]
  {\top \istrue}{}
  \qquad
  \mathrm{(no\ \top E)}
  \qquad
  \mathrm{(no\ \bot I)}
  \qquad
  \infer[\bot E]
  {\Gamma \vdash C \istrue}
  {\Gamma \vdash \bot \istrue}
\]

Negation:

\[
\neg A := A \imp \bot
\]

\section{Examples}

Each of the following formulas $\phi$ can be shown 
$\cdot \vdash \phi \istrue$, i.e. they hold in the empty context, and can
be regarded as ``theorems'' of the system we have defined.

\begin{itemize}
  \item $(A \imp B) \iand A \imp B$
  \item $(A \imp (B \ior C)) \imp (A \iand \neg B) \imp C$
  \item $(A \ior B) \iand C \imp (A \iand C) \ior (B \iand C)$
\end{itemize}


\section{Soundness and Completeness}

What are some properties we believe this logic should have?

What stops us from defining the rules some other way?

What are the ``guiding principles''? What does it even mean to be a logic,
or a logical connective?

We won't be able to answer these questions in full yet, but we can at least
start with one proposed notion of soundness:
{\em It is not possible to derive $\cdot \vdash \bot \istrue$.}

How would we go about demonstrating such a fact?

\section{Counting Proofs}

Here is a related question: how many proofs are there of 

\[
  (A \imp B) \imp (A \iand B) \imp B
\]

?

There are at least two. However, there are in fact infinitely many.



\section{Proof Terms}

For the remaining discussion, it will be convenient to have a more compact
notation for proofs. We therefore introduce {\em proof terms}.

We make a small change to three of our rules. First, for $\imp I$ and $\ior
E$, we add labels to the hypotheses:

\[
  \infer[\imp I^x]
  {\Gamma \vdash A \imp B \istrue}
  {\Gamma, x{:}A\istrue \vdash B \istrue}
\]

\[ 
\infer[\ior E^{x,y}]
  {\Gamma \vdash C \istrue}
  {\Gamma \vdash A \ior B \istrue
   &
   \Gamma, x{:}A\istrue \vdash C \istrue
   &
   \Gamma, y{:}B\istrue \vdash C \istrue
  }
\]

Then, we rename ``the'' hypothesis rule to its corresponding label:

\[
  \infer[x]
  {\Gamma \vdash A \istrue}
  {x{:}A \istrue \in \Gamma}
\]

These changes have the effect of allowing us to refer uniquely to
hypotheses that appear in the context $\Gamma$, and with them we can
abbreviate derivations of the preceding proofs.

In one more step, we apply the following syntactic translation:

\begin{tabular}{cc}
  $\imp I^x(M)$  & $\lambda{x}.\;M$ (function abstraction) \\
  $\imp E(M,N)$  & $M\; N$ (function application) \\
  $\iand I(M,N)$ & $(M, N)$ (pairs) \\
  $\iand E_i(M)$ & $\pi_i M$ (projection) \\
  $\ior I_i(M)$  & $\mathsf{in}_i\,M$ (injection)\\
  $\ior E^{x,y}(M,N,P)$ & $\excase{M, x.\,N, y.\,P}$ (case analysis)\\
  $\top I$ & $()$ (unit)\\
  $\bot E(M)$ & $\mathsf{abort}, M$ (error)
\end{tabular}

This gives us a somewhat familiar-looking syntax akin to functional
programming---more specifically, the simply-typed $\lambda$ calculus
(STLC).


\section{Harmony: Local Soundness and Completeness}

\subsection{(Internal) Soundness and Completeness}

% (TODO additional discussion?)

{\bf Soundness:} the logic is not vacuous. That is,
there are some propositions that are not true.

In particular, we can't write a closed proof of $\ifalse$.

{\bf Completeness:} the hypothetical judgment captures deduction in the logic.
That is, even if we only allow the use of hypotheses
at atomic propositions, then we can still construct
a proof of each proposition by assuming it.

\subsection{Local Soundness as Proof Reduction}

Soundness essentially states that 
a logic's proofs do not admit {\em more information}
than what is used to construct them.
The proof amounts to showing that {\em circuitous}
steps in proofs can be eliminated. 
Soundness is a {\em global} property of the logic:
any rule could interact in some unexpected ways with
all of the other rules, so we can't just check them in isolation.

However, there is a weaker notion of soundness that we can check for just a single
propositional connective: if a proof has circuitousness by virtue of {\em immediately}
introducing and then eliminating a connective, we can eliminate
such a redundancy.
%
We demonstrate this by identifying ``circuitous'' proofs and showing
how they can be rewritten to avoid the unnecessary steps.

Conjunction:

\[
\infer[\iand{E}_1]
{A \istrue}
{
    \infer[\iand{I}]
    {A \iand B \istrue}
    {
        \deduce{A\istrue}{\DD_1}
        &
        \deduce{B\istrue}{\DD_2}
    }
}
\qquad
\Longrightarrow
\qquad
\deduce{A\istrue}{\DD_1}
\]

\[
\infer[\iand{E}_2]
{B \istrue}
{
    \infer[\iand{I}]
    {A \iand B \istrue}
    {
        \deduce{A\istrue}{\DD_1}
        &
        \deduce{B\istrue}{\DD_2}
    }
}
\qquad
\Longrightarrow
\qquad
\deduce{B\istrue}{\DD_2}
\]

Implication:

\[
\infer[\imp{E}]
{B \istrue}
{
\infer[\imp{I}^u]
{A \imp B \istrue}
{   \deduce{B\istrue}
    {\deduce{\vdots}{\infer[u]{A\istrue}{}}}
}
&
\deduce{A}{\DD}
}
\qquad
\Longrightarrow
\qquad
\deduce{B\istrue}
{\deduce{\DD}{\infer[u]{A\istrue}{}}}
\]


\bibliographystyle{plainnat}
\bibliography{main}

\end{document}
